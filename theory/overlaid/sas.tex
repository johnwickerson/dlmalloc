\documentclass[12pt,a4paper]{article}
\usepackage{a4wide}
\usepackage{JohnMath}

\usepackage[svgnames]{xcolor}

% CODE LISTINGS
\usepackage{listings}
\lstset{
  language=C,
  columns=[l]fullflexible,
  mathescape=true,
  basicstyle=\ttfamily\color{Black},
  showstringspaces=false,
  commentstyle=\color{DarkGreen}, 
  numbers=none, 
  escapechar=£,
  escapebegin=\normalsize\rmfamily\color{Black}
}

% SPECIFICATIONS
\newcommand{\ml}[2][t]{\mbox{\mdseries\begin{tabular}[#1]{@{}L@{}}#2\end{tabular}}}
\newcommand{\ass}[1]{\ensuremath{{\color{blue}\left\{\ml[c]{#1}\right\}}}}
\newcommand{\seqspec}[3]{\ass{#1}\,{\mbox{{\tt #2}}}\,\ass{#3}}
\newcommand{\Seqspec}[3]{\multicolumn{2}{l}{$\ass{#1}$ {#2} $\ass{#3}$}}
\newcommand{\comm}[1]{\vspace{-2pt}%
    \begin{list}{/$*$}{%
        \topsep=5pt%
        \leftmargin=3cm%
      }\item #1 \hfill$*$/\end{list}%
}

\renewcommand{\arraystretch}{1.2}

\renewcommand{\true}{\mathsf{tt}}
\renewcommand{\emp}{\emph{emp}}

\title{Separation Algebras}
\author{John Wickerson}
\date{}

\begin{document}

\maketitle

A separation algebra (SA) is a structure $(M, ∘, U)$ such that for some $\top ∉ M$:
\begin{itemize}
\item $∘ ∈ M^\top × M^\top → M^\top$
\item $U⊆M$
\item $∀x∈M^\top.\top ∘ x = \top$ 
\item $∀m∈M.∃u∈U.u ∘ m = m$
\item $∀x,y,z∈M^\top.(x ∘ y) ∘ z = x ∘ (y ∘ z)$
\item $∀x,y∈M^\top.x ∘ y = y ∘ z$
\item $∀x_1,x_2,y∈M^\top.x_1 ∘ y = x_2 ∘ y ≠ \top ⇒ x_1 = x_2$
\end{itemize}
Adapted from Dockins et al.:

\begin{defn}
Given a set $M$ we can construct the flat SA $M_\bot$ like so:
\[
M_\bot \eqdef (M\uplus\{\bot\}, ∘, \{\bot\})
\]
where:
\[
m ∘ m \eqdef \begin{cases} m'  & \text{if $m = \bot$} \\ m  & \text{if $m' = \bot$} \\ \top & \text{otherwise} \end{cases}
\] 
\end{defn}

\begin{defn}
Given an SA $S = (M, ∘_S, U)$, we can construct the discrete SA $S_=$ like so:
\[
S_= \eqdef (M, ∘, M)
\]
where
\[
m ∘_= m' \eqdef \begin{cases} m  & \text{if $m = m'$} \\ \top & \text{otherwise} \end{cases}
\] 
\end{defn}

\begin{defn}
Given SAs $S_1 = (M, ∘_1, U_1)$ and $S_2 = (N, ∘_2, U_2)$ we can construct the product SA $S_1 × S_2$ like so:
\[
S_1 × S_2 \eqdef (M×N, ∘, U_1 × U_2)
\]
where
\[
(m,n) ∘ (m',n') \eqdef \begin{cases} (m ∘_1 m', n ∘_2 n') & \text{if $m ∘_1 m' ≠ \top$ and $n ∘_2 n' ≠ \top$} \\ \top & \text{otherwise} \end{cases}
\]
\end{defn}

\begin{defn}
Given a set $M$ and an SA $S = (N, ∘_S, U_S)$ we can construct the finite function SA $M →_{\rm fin} S = (M→_{\rm fin}N, ∘, U)$, where
\[
\begin{array}{rcl}
M →_{\rm fin} N &\eqdef& \{f ∈ M→N \mid \mathop{\rm finite}(f^{-1}\, U_N)\} \\
f_1 ∘ f_2 &\eqdef& \begin{cases} λm∈M.f_1\,m ∘_S f_2\,m & \text{if $∀m∈M.f_1\,m ∘_S f_2\,m ≠ \top$} \\ \top & \text{otherwise} \end{cases} \\
U &\eqdef& \{f\mid ∀m∈M.f\,m ∈ U_S\}
\end{array}
\]
\end{defn}

\noindent A state is an element of a finite function SA from naturals to the flat SA of integers. A world is an element of the product SA comprising a state together with a discretised finite function SA from region names to states. 
\[
\begin{array}{lclcl}
l,s &∈& {\sf State} &\eqdef& \mathbb N →_{\rm fin} \mathbb Z_{\bot} \\
w &∈& {\sf World} &\eqdef& {\sf State} × ({\sf RName} →_{\rm fin} {\sf State})_=
\end{array}
\]

\begin{remark}
This isn't quite right -- in fact the local state and each of the shared states must be pairwise disjoint.
\end{remark}

\end{document}