\documentclass[12pt,a4paper]{article}
\usepackage{a4wide}
\usepackage{JohnMath}

\usepackage[svgnames]{xcolor}

% CODE LISTINGS
\usepackage{listings}
\lstset{
  language=C,
  columns=[l]fullflexible,
  mathescape=true,
  basicstyle=\ttfamily\color{Black},
  showstringspaces=false,
  commentstyle=\color{DarkGreen}, 
  numbers=none, 
  escapechar=£,
  escapebegin=\normalsize\rmfamily\color{Black}
}

% SPECIFICATIONS
\newcommand{\ml}[2][t]{\mbox{\mdseries\begin{tabular}[#1]{@{}L@{}}#2\end{tabular}}}
\newcommand{\ass}[1]{\ensuremath{{\color{blue}\left\{\ml[c]{#1}\right\}}}}
\newcommand{\seqspec}[3]{\ass{#1}\,{\mbox{{\tt #2}}}\,\ass{#3}}
\newcommand{\Seqspec}[3]{\multicolumn{2}{l}{$\ass{#1}$ {#2} $\ass{#3}$}}
\newcommand{\comm}[1]{\vspace{-2pt}%
    \begin{list}{/$*$}{%
        \topsep=5pt%
        \leftmargin=3cm%
      }\item #1 \hfill$*$/\end{list}%
}

\renewcommand{\arraystretch}{1.2}

\renewcommand{\true}{\mathsf{tt}}
\renewcommand{\emp}{\emph{emp}}

%\renewcommand{\boxed}[2][]{{\textbf{[}}#2{\textbf{]}}_{#1}}

\newcommand{\rsem}[1]{{(\![}{#1}{]\!)}}

\newcommand{\starstar}{\mathbin{\bar{*}}}
\newcommand{\compcomp}{\mathbin{\bar{·}}}

\newcommand{\defined}{\mathop{\text{defined}}}

\newcommand{\SET}[2]{\left\{\begin{array}{@{}l|l@{}} #1 & #2 \end{array}\right\}}

\newenvironment{mapping}{\left\{ \begin{array}{@{}r@{\,↦\,}l@{}}}{\end{array}\right\}}

\title{GSep}
\author{John Wickerson}
\date{}

\begin{document}

\maketitle

\section{Preliminaries}

\subsection{Spatial closure operators}
Suppose $R$ and $S$ are of type $\emph{expr} → \emph{expr} → \emph{assertion}$. Define:
\[
\begin{array}{rcl}
R;S &\eqdef& λx\,z.∃y.R\,x\,y ∧ S\,y\,z \\
R ∨ S &\eqdef& λx\,y. R\,x\,y ∨ S\,x\,y \\
\id &\eqdef& λx\, y
.x=y ∧ \emp
\end{array}
\]

\noindent Then let $R^*$ be the least function satisfying $R^* = \id ∨ (R;R^*)$ and let $R^+ = R;R^*$.



\section{GSep}

GSep is RGSep without the R!

\subsection{States and worlds}

A state is a finite partial function from naturals to integers. A segmented state is a total mapping from segment names to states, of which a finite number are non-empty. A world comprises a `local' state paired with a segmented state. The local state and all the shared states are pairwise disjoint. 
\[
\begin{array}{rclcl}
l,s,σ &∈& {\sf State} &\eqdef& \mathbb N \rightharpoonup_{\rm fin} \mathbb Z \\
w &∈& {\sf World} &\eqdef& \{\ang{l,S} ∈ {\sf State} × ({\sf Seg}→_{\rm fin} {\sf State}) \mid \defined(l · ⊙S) \}
\end{array}
\]

\noindent For states, $s·s'$ is defined as $s\uplus s'$ when $\dom\,s ∩ \dom\,s' = ∅$. We write $\odot$ for iterated~$·$. We write $⊙S$ as shorthand for $⊙_a\,S\,a$. For worlds, $\ang{l,S} · \ang{l',S'}$ is defined as $\ang{l·l',S}$ when $l·l'$ is defined and $S=S'$. We lift $·$ to $*$ in the usual way.


\subsection{Assertions}

The syntax of GSep assertions is as follows, where $p$ is an ordinary separation logic assertion (a set of {\sf State}s):
\[
\begin{array}{r@{\ }l}
P ::= \boxed[a]{P} \mid \N a{:}A.P \mid \N (a{:}A, b{:}B).P \mid P * P \mid P ∨ P \mid ∃x.P \mid p
\end{array}
\]
\begin{remark}
Should include abstract predicates in the syntax too.
\end{remark}

\noindent The semantics of an assertion is a set of worlds; that is:
\[
\sem{P} : \pow({\sf World})
\]
We define the semantics of assertions as follows.
\[
\begin{array}{rcl}
\sem{p} &≝& \{\ang{l,S}\mid l ⊧ p\} \\
\sem{P_0 * P_1} &≝& \sem{P_0} * \sem{P_1} \\
\sem{P_0 ∨ P_1} &≝& \sem{P_0} ∪ \sem{P_1} \\
\sem{∃x.P} &≝& \bigcup_v.\sem{P[v/x]} \\
\sem{\,\boxed[a]{P}\,} &≝& \{\ang{∅,S} \mid \ang{S\,a, S[a:=∅]} ∈ \sem{P}\}\\
\sem{\N a{:}A.P} &≝& \{\ang{l · S\,a, S[a:=∅]}\mid \ang{l,S} ∈\sem{P}\} \\
\sem{\N (a{:}A, b{:}B).P} &≝& \{\ang{l · S\,a · S\,b, S[a:=∅,b:=∅]}\mid \ang{l,S} ∈\sem{P}\}
\end{array}
\]

\subsection{Actions}

Syntactically, an action is either a basic action `$G \mid P \rightsquigarrow Q \mid R$', or a finite union thereof. Semantically, an action denotes a set of pairs of worlds that can arise from transforming a given segment $a$ in accordance with the action's specification. Notably, the action is free to transform the contents of other segments without restriction.
\[
\sem{a : A} : \pow({\sf World}^2)
\]

\noindent Consider firing the basic action `$G \mid P \rightsquigarrow Q \mid R$' on region $a$. The effect of the action is to allow any transformation to the world in which a part of segment $a$ satisfying $P$ has been replaced by a part satisf replaces a part of a given segment, say $a$, that satisfies $P$ with a part satisfying $Q$, provided $R$ also holds in $a$, and $G$ holds of the local state. Segments other than $a$ can be modified arbitrarily. We provide the following long-hand form:
\[
\begin{array}{l}
{\sf action} A \\
\left\lfloor\begin{array}{ll}
{\sf pre} & P \\
{\sf post} & Q \\
{\sf context} & R \\
{\sf guard} & G
\end{array}\right.
\end{array}
\]

\noindent This is its semantics:
\[
\sem{a:A} \eqdef \SET{\ml{(\ang{l,S[a:=σ·σ'']}, \\ \ang{l',S'[a:=σ'·σ'']})}}{\ml{\ang{σ,S[a:=∅]} ∈ \sem{P} ∧ \ang{σ',S'[a:=∅]}∈\sem{Q} ∧ {}\\ \ang{σ'',S[a:=∅]} ∈ \sem{R * \true} ∧ \ang{l,S[a:=∅]} ∈ \sem{G*\true}}}
\]

\noindent We allow actions to be formed by the union of existing actions, like so:
\[
\sem{a : A_1 ∪ A_2} \eqdef \sem{a : A_1} ∪ \sem{a : A_2}
\]

\subsection{Stability}

\newcommand{\stable}[1][]{\mathrel{\text{stable}_{#1}}}

Define `$⊧ P\stable[a] A$' to hold iff whenever $\ang{l,S} ∈\sem{P}$ and $(\ang{o,S},\ang{o',S'})∈\sem{a: A}$ and $\defined(l·\odot S·o)$ and $\defined(l·\odot S')$ then $\ang{l,S'}∈\sem{P}$. Define `$⊧ P\stable Γ$' to hold iff $⊧ P\stable[a]A$ holds for all $(a:A) ∈ Γ$. Here are some (unchecked) proof rules for reasoning about stability.
\begin{mathpar}
\inferrule{⊦ P\stable Γ \\ ⊦ Q\stable Γ}
{⊦ P * Q\stable Γ}
\and
\inferrule{⊦ P\stable Γ \\ ⊦ Q\stable Γ}
{⊦ P ∧ Q\stable Γ}
\and
\inferrule{⊦ P\stable Γ \\ ⊦ Q\stable Γ}
{⊦ P ∨ Q\stable Γ}
\and
\inferrule{(a:A)∈Γ \\ ⊦ P\stable[a]A}
{⊦ \boxed[a]{P}\stable Γ}
\and
\inferrule{ }{⊦ p\stable Γ}
\and
\inferrule{⊦ P\stable Γ}{⊦ \N a{:}A.P\stable Γ[a:A]}
\end{mathpar}


\subsection{GSep quadruples}

First define $\sem{Γ} \eqdef \bigcap_{(a:A)∈ Γ}\,\sem{a: A}$. Then the semantics of the GSep quadruple `$Γ⊧\{P\}\,C\,\{Q\}$' is as follows:
\[
\ml{∀l\,S\,σ\,σ'.(l·⊙S) = σ ∧ \ang{l,S}∈\sem{P} ∧ (C,σ)⇓σ' \\ {} \qquad\qquad\qquad {} ⇒ ∃l'\,S'.σ' = (l'·⊙S') ∧ \ang{l',S'}∈\sem{Q} ∧ (\ang{l,S},\ang{l',S'})∈ \sem{Γ}}
\]

\subsection{Some GSep rules}

Here is the frame rule: 
\[
\inferrule*[right=Frame]
{
Γ ⊦ \left\{P\right\}\,C\,\left\{Q\right\}
\\
R\stable Γ}
{Γ ⊦ \left\{P * R\right\}\,C\,\left\{Q * R\right\}}
\]

\noindent Here is a rule for reading from a region (taken from Alias Logic).
\[
\inferrule*[right=RegRead]{ }
{Γ[a{:}A] ⊦ \left\{\boxed[a]{e ↦ e'  *  P[e'/x]}\right\}\, x\texttt{:=[}e\texttt{]}\, \left\{\boxed[a]{e ↦ e'  *  P}\right\}}
\]

\noindent Here is a rule for hiding a region.
\[
\inferrule*[right=Hide]
{
Γ[a{:}A] ⊦ \left\{P\right\}\,C\,\left\{Q\right\}
\\
a ∉ \dom Γ}
{Γ ⊦ \left\{\N a{:}A.P\right\}\,C\,\left\{\N a{:}A.Q\right\}}
\]

\noindent Here is a rule for weakening the guarantee.
\[
\inferrule*[right=GuarWeaken]
{
Γ' ⊦ \left\{P\right\}\,C\,\left\{Q\right\}
\\
\sem{Γ'} ⊆ \sem{Γ}}
{Γ ⊦ \left\{P\right\}\,C\,\left\{Q\right\}}
\]

\noindent Here is a rule for hiding two regions simultaneously.
\[
\inferrule*[right=TwoHide]
{
Γ[a{:}A, b{:}B] ⊦ \left\{P\right\}\,C\,\left\{Q\right\}
\\
\textstyle{a,b ∉ \domΓ}}
{Γ ⊦ \left\{\N(a{:}A,b{:}B).P\right\}\,C\,\left\{\N(a{:}A,b{:}B).Q\right\}}
\]

\noindent Here is a rule for updating a region. Note that we `nullify' the value of $a$ in $Γ$, rather than removing the mapping altogether, because $P$ and $Q$ might mention it. 
\[
\inferrule*[right=RegUpd]{
Γ[a{:}∅] ⊦ \left\{P' * P\right\}\,C\,\left\{Q' * Q\right\}
\\
P,Q \text{ precise}
\\
\sem{a : P' \mid P \rightsquigarrow Q \mid R}  ⊆  \sem{a : A} 
}
{Γ[a{:}A] ⊦ \left\{P' * \boxed[a]{P*R}\right\}\,C\,\left\{Q' * \boxed[a]{Q*R}\right\}}
\]

\noindent Here is a rule for updating two regions simultaneously.
\[
\inferrule*[right=TwoRegUpd]{
Γ[a{:}∅, b{:}∅] ⊦ \left\{P' * P_1 * P_2\right\}\,C\,\left\{Q' * Q_1 * Q_2\right\}
\\
P_1, Q_1, P_2, Q_2 \text{ precise}
\\
\sem{a : P' \mid P_1 \rightsquigarrow Q_1 \mid R_1}  ∩  \sem{b : P' \mid P_2 \rightsquigarrow Q_2 \mid R_2} ⊆  \sem{a : A} ∩ \sem{b : B}}
{Γ[a{:}A, b{:}B] ⊦ \left\{P'  *  \boxed[a]{P_1 * R_1}  *  \boxed[b]{P_2 * R_2}\right\}\,C\,\left\{Q'  *  \boxed[a]{Q_1 * R_1}  *  \boxed[b]{Q_2 * R_2}\right\}}
\]

\begin{remark}
In the {\sc RegUpd} and {\sc TwoRegUpd} rules, we may need to intersect with a relation that prohibits changes to other regions.
\end{remark}


\section{A singly-indexed list}

Our first datastructure is a singly-indexed list. Every node has a value and a pointer to the next node. The final node's next pointer is set to 0. The first node is a sentinel, at a fixed location $r$. Our datastructure can be described by the following formulae:
\[
\begin{array}{rcl}
\emph{list}\,r &\iffdef & el^+\,r\,0 \\
x ∈ \emph{list}\,r &\iffdef& el^+\,r\,x  *  el^+\,x\,0
\end{array}
\]
where:
\[
el\,x\,y  =  x^1 \pto[.5] \underscore  *  x^2↦y
\]

\noindent Our datastructure provides two methods: insertion and deletion. These are implemented as follows.

\begin{lstlisting}
insert(x){
  int* p = r;
  while ([p+1]!=0 && ...) do p:=[p+1];
  [x+1]:=[p+1];
  [p+1]:=x;
}
\end{lstlisting}

\begin{lstlisting}
remove(x){
  int* p = r;
  while ([p+1]!=x) do p:=[p+1];
  [p+1]:=[x+1];
}
\end{lstlisting}

\noindent We can specify these methods like so:
\[
\begin{array}{c}
c{:}C ⊦ \seqspec{\boxed[c]{\emph{list}\,{\tt r}}  *  {\tt x}^1 ↦ v  *  {\tt x}^2 ↦ \underscore}{insert(x)}{\boxed[c]{{\tt x}∈\emph{list}\,{\tt r}}  *  {\tt x}^1\pto[.5]v} \\
c{:}C ⊦ \seqspec{\boxed[c]{{\tt x}∈\emph{list}\,{\tt r}}  *  {\tt x}^1\pto[.5]v}{remove(x)}{\boxed[c]{\emph{list}\,{\tt r}}  *  {\tt x}^1 ↦ v  *  {\tt x}^2 ↦ \underscore}
\end{array}
\]

\noindent The region $c$ denotes the module's internal state. Its interference, $C$, is as follows:
\[
\begin{array}{rcl}
C &\eqdef& \bigcup_{x∈\mathbb N}\,\{\textsc{Add}\,x, \textsc{Rm}\,x\}
\end{array}
\]

\noindent where:
\[
\begin{array}{l}
{\sf action} \textsc{Add}\,x \\
\left\lfloor\begin{array}{ll}
{\sf pre} & p^2 ↦ p' \\
{\sf post} & p^2 ↦ x * x^1\pto[.5] v * x^2↦p' \\
{\sf context} & el^*\,r\,p \\
\end{array}\right. \\ \\
{\sf action} \textsc{Rm}\,x \\
\left\lfloor\begin{array}{ll}
{\sf pre} & p^2 ↦ x * x^1\pto[.5] v * x^2↦p' \\
{\sf post} & p^2 ↦ p' \\
{\sf context} & el^*\,r\,p \\
{\sf guard} & x^1\pto[.5]\underscore
\end{array}\right.
\end{array}
\]

\noindent It is important to note that the pre- and post-conditions of {\tt insert} and {\tt remove} are stable under $C$. 

\subsection{Verification of the {\tt insert} method}

\begin{lstlisting}
insert(x){
  £\ass{\boxed[c]{\emph{list}\,{\tt r}} * {\tt x}^1 ↦ v * {\tt x}^2↦\underscore}£
  // begin frame
    £\ass{\boxed[c]{\emph{list}\,{\tt r}}}£
    £\ass{\boxed[c]{el^+\,{\tt r}\,0}}£
    £\ass{\boxed[c]{el^*\,{\tt r}\,{\tt r}  *  el^+\,{\tt r}\,0}}£
    int* p = r; // using Hoare's assignment axiom
    £\ass{\boxed[c]{el^*\,{\tt r}\,{\tt p} * el^+\,{\tt p}\,0}}£
    £\ass{∃p'.\boxed[c]{el^*\,{\tt r}\,{\tt p} * el\,{\tt p}\,p' * el^*\,p'\,0}}£
    £\ass{∃p'.\boxed[c]{el^*\,{\tt r}\,{\tt p} * {\tt p}^1\pto[.5]\underscore  *  {\tt p}^2↦p' * el^*\,p'\,0}}£
    // begin existential
      £\ass{\boxed[c]{el^*\,{\tt r}\,{\tt p} * {\tt p}^1\pto[.5]\underscore  *  {\tt p}^2↦p' * el^*\,p'\,0}}£
      int* t = [p+1]; // using RegRead axiom
      £\ass{\boxed[c]{el^*\,{\tt r}\,{\tt p} * {\tt p}^1\pto[.5]\underscore  *  {\tt p}^2↦{\tt t} * el^*\,{\tt t}\,0}}£
    // end existential
    £\ass{∃p'.\boxed[c]{el^*\,{\tt r}\,{\tt p} * {\tt p}^1\pto[.5]\underscore  *  {\tt p}^2↦{\tt t} * el^*\,{\tt t}\,0}}£
    £\ass{\boxed[c]{el^*\,{\tt r}\,{\tt p} * {\tt p}^1\pto[.5]\underscore  *  {\tt p}^2↦{\tt t} * el^*\,{\tt t}\,0}}£
    £\ass{\boxed[c]{el^*\,{\tt r}\,{\tt p} * el\,{\tt p}\,{\tt t} * el^*\,{\tt t}\,0}}£
    while (t!=0 && ...) do {
      £\ass{\boxed[c]{el^*\,{\tt r}\,{\tt p} * el\,{\tt p}\,{\tt t} * el^*\,{\tt t}\,0}  *  {\tt t}\dot{≠} 0}£
      £\ass{\boxed[c]{el^*\,{\tt r}\,{\tt t} * el^*\,{\tt t}\,0}  *  {\tt t}\dot{≠} 0}£
      £\ass{∃t'.\boxed[c]{el^*\,{\tt r}\,{\tt t} *  {\tt t}^1\pto[.5]\underscore  *  {\tt t}^2↦t'  * el^*\,t'\,0}}£
      p:=t; // using Hoare's assignment axiom
      £\ass{∃t'.\boxed[c]{el^*\,{\tt r}\,{\tt p} *  {\tt p}^1\pto[.5]\underscore  *  {\tt p}^2↦t'  * el^*\,t'\,0}}£
      // begin existential
        £\ass{\boxed[c]{el^*\,{\tt r}\,{\tt p} *  {\tt p}^1\pto[.5]\underscore  *  {\tt p}^2↦t'  * el^*\,t'\,0}}£
        t:=[p+1]; // using RegRead axiom
        £\ass{\boxed[c]{el^*\,{\tt r}\,{\tt p} *  {\tt p}^1\pto[.5]\underscore  *  {\tt p}^2↦{\tt t}  * el^*\,{\tt t}\,0}}£
      // end existential
      £\ass{∃t'.\boxed[c]{el^*\,{\tt r}\,{\tt p} *  {\tt p}^1\pto[.5]\underscore  *  {\tt p}^2↦{\tt t}  * el^*\,{\tt t}\,0}}£
      £\ass{\boxed[c]{el^*\,{\tt r}\,{\tt p} *  {\tt p}^1\pto[.5]\underscore  *  {\tt p}^2↦{\tt t}  * el^*\,{\tt t}\,0}}£
    }
  // end frame
  £\ass{\boxed[c]{el^*\,{\tt r}\,{\tt p} * {\tt p}^1 \pto[.5]\underscore  * {\tt p}^2↦{\tt t} * el^*\,{\tt t}\,0}  * {\tt x}^1 ↦ v * {\tt x}^2↦\underscore}£
  // begin null action on region £$c$£
    £\ass{{\tt x}^1 ↦ v * {\tt x}^2↦\underscore}£
    [x+1]:=t;
    £\ass{{\tt x}^1 ↦ v * {\tt x}^2↦{\tt t}}£
  // end null action
  £\ass{\boxed[c]{el^*\,{\tt r}\,{\tt p} * {\tt p}^1 \pto[.5]\underscore  * {\tt p}^2↦{\tt t} * el^*\,{\tt t}\,0}  * {\tt x}^1 ↦ v * {\tt x}^2↦{\tt t}}£
  // begin action £$\textsc{Add}\,{\tt x}$£ on region £$c$£
    £\ass{{\tt p}^2↦{\tt t}  * {\tt x}^1 ↦ v * {\tt x}^2↦{\tt t}}£
    [p+1]:=x;
    £\ass{{\tt p}^2↦{\tt x}  * {\tt x}^1 \pto[.5] v * {\tt x}^2↦{\tt t}  * {\tt x}^1 \pto[.5] v}£
  // end action
  £\ass{\boxed[c]{el^*\,{\tt r}\,{\tt p} * {\tt p}^1 \pto[.5]\underscore  * {\tt p}^2↦{\tt x}  * {\tt x}^1 \pto[.5] v * {\tt x}^2↦{\tt t}  * el^*\,{\tt t}\,0}  * {\tt x}^1 \pto[.5] v}£
  £\ass{\boxed[c]{el^+\,{\tt r}\,{\tt x} * el^+\,{\tt x}\,0}  *  {\tt x}^1\pto[.5]v}£
  £\ass{\boxed[c]{{\tt x}∈\emph{list}\,{\tt r}}  *  {\tt x}^1\pto[.5]v}£
}

\end{lstlisting}


\section{A doubly-indexed list}

Let us move to a doubly-indexed list. Every node now has three fields: a value, and two next pointers. The two chains of next pointers present two orderings on the same set of nodes. Both orderings begin at the same sentinel node, which at a fixed location $r$. Our datastructure can be described by the following formulae:
\[
\begin{array}{rcl}
\emph{2list}\,r &\iffdef& el_0^+\,r\,0  ∧  el_1^+\,r\,0 \\
x ∈ \emph{2list}\,r &\iffdef& (el_0^+\,r\,x  *  el_0^+\,x\,0)  ∧  (el_1^+\,r\,x  *  el_1^+\,x\,0)
\end{array}
\]
where:
\[
\begin{array}{rcl}
el_0\,x\,y &\eqdef& x^1 \pto[.5] \underscore  *  x^2↦y  *  x^3 ↦\underscore \\
el_1\,x\,y &\eqdef& x^1 \pto[.5] \underscore  *  x^2↦\underscore *  x^3 ↦y
\end{array}
\]

\noindent The implementations of the insert and remove methods become:

\begin{lstlisting}
insert(x){
  int* p = r;
  int* q = r;
  while ([p+1]!=0 && ...) do p:=[p+1];
  while ([q+2]!=0 && ...) do q:=[q+2];
  [x+1]:=[p+1];
  [x+2]:=[q+2];
  [p+1]:=x;
  [q+2]:=x;
}
\end{lstlisting}

\begin{lstlisting}
remove(x){
  int* p = r;
  int* q = r;
  while ([p+1]!=x) do p:=[p+1];
  while ([q+2]!=x) do q:=[q+2];
  [p+1]:=[x+1];
  [q+2]:=[x+2];
}
\end{lstlisting}

\noindent The specifications become:
\[
\begin{array}{c}
c{:}C ⊦ \seqspec{\boxed[c]{\emph{2list}\,{\tt r}}  *  {\tt x}^1 ↦ v  *  {\tt x}^2 ↦ \underscore  *  {\tt x}^3 ↦ \underscore}{insert(x)}{\boxed[c]{{\tt x}∈\emph{2list}\,{\tt r}}  *  {\tt x}^1\pto[.5]v} \\
c{:}C ⊦ \seqspec{\boxed[c]{{\tt x}∈\emph{2list}\,{\tt r}}  *  {\tt x}^1\pto[.5]v}{remove(x)}{\boxed[c]{\emph{2list}\,{\tt r}}  *  {\tt x}^1 ↦ v  *  {\tt x}^2 ↦ \underscore  *  {\tt x}^3 ↦ \underscore}
\end{array}
\]

\noindent One difficulty arises when we try to specify the module's internal interference $C$. Suppose we wish to add a new element after element $p$ in the first list and after element $q$ in the second. The context for this action should require that $p$ and $q$ are indeed elements in the respective lists; that is, both $el_0^*\,{\tt r}\,p$ and $el_1^*\,{\tt r}\,q$ should hold. However, these assertions may overlap, so we can't combine them with the $*$-operator. Nor do they completely overlap, so we can't use the $∧$-operator either. And we can't write  $(el_0^*\,{\tt r}\,p * \true) ∧ (el_1^*\,{\tt r}\,q * \true)$ either, because this assertion may include the elements $p$ and $q$ that we wish to mutate!

Another difficulty arises while verifying the {\tt insert} method, during the following step:
\begin{lstlisting}
£\ass{(el_0^*\,{\tt r}\,{\tt p} * el_0\,{\tt p}\,p' *  el_0\,{\tt x}\,p'  *  el_0^*\,p'\,0) ∧ (el_1^*\,{\tt r}\,{\tt q} * el_1\,{\tt q}\,q'  *  el_1\,{\tt x}\,q'  *  el_1^*\,q'\,0)}£
[p+1]:=x;
[q+2]:=x;
£\ass{(el_0^*\,{\tt r}\,{\tt p} * el_0\,{\tt p}\,{\tt x} *  el_0\,{\tt x}\,p'  *  el_0^*\,p'\,0) ∧ (el_1^*\,{\tt r}\,{\tt q} * el_1\,{\tt q}\,{\tt x}  *  el_1\,{\tt x}\,q'  *  el_1^*\,q'\,0)}£
\end{lstlisting}

\noindent We can deduce this using the conjunction rule. One antecedant is the following:
\begin{lstlisting}
£\ass{el_0^*\,{\tt r}\,{\tt p} * el_0\,{\tt p}\,p' *  el_0\,{\tt x}\,p'  *  el_0^*\,p'\,0}£
[p+1]:=x;
[q+2]:=x;
£\ass{el_0^*\,{\tt r}\,{\tt p} * el_0\,{\tt p}\,{\tt x} *  el_0\,{\tt x}\,p'  *  el_0^*\,p'\,0}£
\end{lstlisting}

\noindent The other antecedant is similar. The task then reduces to proving:
\[
\seqspec{el_0^*\,{\tt r}\,{\tt p} * el_0\,{\tt p}\,{\tt x} *  el_0\,{\tt x}\,p'  *  el_0^*\,p'\,0}{[q+2]:=x}{el_0^*\,{\tt r}\,{\tt p} * el_0\,{\tt p}\,{\tt x} *  el_0\,{\tt x}\,p'  *  el_0^*\,p'\,0}
\]

\noindent Now intuitively, this specification should hold; the $el_0$ predicate is sensitive only to the values of the \emph{second} fields of nodes, so updating the \emph{third} field of node {\tt q} should preserve the predicate. But we are making assumptions about {\tt q}; namely, that it corresponds to a valid node in the list. For if {\tt q} is an arbitrary heap address, then its third field, located at ${\tt q}+2$, could coincide with the second field of some node in the list, and in this case, the postcondition would fail. Of course, we \emph{are} justified in making this assumption about {\tt q}, because we know that $el_1^*\,{\tt r}\,{\tt q}$ holds. Unfortunately, we deleted this fact when we applied the conjunction rule. And it's not clear how we could have kept that fact around.

We could avoid this hassle by assuming list nodes to be nicely aligned in memory -- that is, that the address of the first cell of any node is divisible by 3. However, we don't actually want to make this restriction. Morally speaking, we should be able to do without it, and practically speaking, dlmalloc's structures are by no means `nicely aligned'.


\section{Co-referring regions}

\newcommand{\elhat}[2]{\widehat{el}_{#1, #2}}
\newcommand{\Add}[1]{\textsc{Add}_{#1}}
\newcommand{\Rm}[1]{\textsc{Rm}_{#1}}


We propose to describe the datastructure in a very different way. We shall see it as two \emph{separate} lists (that is, we will use separating conjunction where previously we had ordinary conjunction). But in order to preserve the close relationship between the two lists (namely, that every node appearing in one list also appears in the other) we shall use `ghost pointers', which map each element of one list to its counterpart in the other list. Here is our first attempt:
\[
\begin{array}{rcl}
\emph{2list}\, r &\iffdef& \N a.\N b.\boxed[a]{\elhat br^+\,r\,0} *  \boxed[b]{\elhat ar^+\,r\,0} \\
x∈\emph{2list}\,r &\iffdef& \N a.\N b.\boxed[a]{\elhat br^+\,r\,x  *  \elhat br^+\,x\,0} *  \boxed[b]{\elhat ar^+\,r\,x  *  \elhat ar^+\,x\,0}
\end{array}
\]
where:
\[
\begin{array}{rcl}
el_0\,x\,y &\iffdef& x^1 \pto[.25] \underscore  *  x^2↦y \\
el_1\,x\,y &\iffdef& x^1 \pto[.25] \underscore  *  x^3↦y
\end{array}
\]
and:
\[
\begin{array}{rcl}
in_{a,r}\,x &\iffdef& \boxed[a]{el_0^*\,r\,x  *  \true} \\
in_{b,r}\,x &\iffdef& \boxed[b]{el_1^*\,r\,x  *  \true}
\end{array}
\]
and:
\[
\begin{array}{rcl}
\elhat br\,x\,y &\iffdef& el_0\,x\,y  *  in_{b,r}\,x \\
\elhat ar\,x\,y &\iffdef& el_1\,x\,y  *  in_{a,r}\,x
\end{array}
\]

\noindent The predicate $\elhat br$ describes an element that appears in the first list. It uses the $in_{b,r}$ predicate to specify that the element appears in the second list too. From this and the symmetric fact about $\elhat ar$, we can deduce that the two lists pass through exactly the same set of elements.

We specify the insert and remove methods in the same way as before (but now with the new implementation of the $\emph{2list}$ predicate):
\[
\begin{array}{c}
c{:}C_{\tt r} ⊦ \seqspec{\boxed[c]{\emph{2list}\,{\tt r}}  *  {\tt x}^1 ↦ v  *  {\tt x}^2 ↦ \underscore  *  {\tt x}^3 ↦ \underscore}{insert(x)}{\boxed[c]{{\tt x}∈\emph{2list}\,{\tt r}}  *  {\tt x}^1\pto[.5]v} \\
c{:}C_{\tt r} ⊦ \seqspec{\boxed[c]{{\tt x}∈\emph{2list}\,{\tt r}}  *  {\tt x}^1\pto[.5]v}{remove(x)}{\boxed[c]{\emph{2list}\,{\tt r}}  *  {\tt x}^1 ↦ v  *  {\tt x}^2 ↦ \underscore  *  {\tt x}^3 ↦ \underscore}
\end{array}
\]

\noindent where
\[
C_r \eqdef \bigcup_{x∈\mathbb N}\,\{\Add{r}\,x, \Rm{r}\,x\}
\]

\noindent and
\[
\begin{array}{l}
{\sf action} \Add r\,x \\
\left\lfloor\begin{array}{ll}
{\sf pre} & p^2 ↦ p'  *  q^3↦q' \\
{\sf post} & p^2 ↦ x  *  q^3 ↦ x  * x^1\pto[.5] v  * x^2↦p'  *  x^3 ↦ q' \\
{\sf context} & el_0^*\,r\,p  *  el_1^*\,r\,q
\end{array}\right. \\ \\
{\sf action} \Rm r\,x \\
\left\lfloor\begin{array}{ll}
{\sf pre} & p^2 ↦ x  *  q^3 ↦ x  * x^1\pto[.5] v * x^2↦p'  *  x^3 ↦ q' \\
{\sf post} & p^2 ↦ p'  *  q^3↦q' \\
{\sf context} & el_0^*\,r\,p  *  el_1^*\,r\,q \\
{\sf guard} & x^1\pto[.5]v
\end{array}\right.
\end{array}
\]

\subsection{Verifying the {\tt insert} method}

\begin{lstlisting}
£\ass{\boxed[c]{\emph{2list}\,{\tt r}}  *  {\tt x}^1 ↦ v  *  {\tt x}^2 ↦ \underscore  *  {\tt x}^3 ↦ \underscore}£
insert(x){
  // begin frame
    £\ass{\boxed[c]{\emph{2list}\,{\tt r}}  *  {\tt x}^1 \pto[.5] v  *  {\tt x}^2 ↦ \underscore  *  {\tt x}^3 ↦ \underscore}£
    // begin frame
      £\ass{\boxed[c]{\emph{2list}\,{\tt r}}}£
      £\ass{\boxed[c]{\N a.\N b.\boxed[a]{\elhat b{\tt r}^+\,{\tt r}\,0}  *  \boxed[b]{\elhat a{\tt r}^+\,{\tt r}\,0}}}£
      £\ass{\boxed[c]{\N a.\N b.\boxed[a]{\elhat b{\tt r}^*\,{\tt r}\,{\tt r}  *  \elhat b{\tt r}^+\,{\tt r}\,0}  *  \boxed[b]{\elhat a{\tt r}^*\,{\tt r}\,{\tt r}  *  \elhat a{\tt r}^+\,{\tt r}\,0}}}£
      int* p = r;
      int* q = r;
      £\ass{\boxed[c]{\N a.\N b.\boxed[a]{\elhat b{\tt r}^*\,{\tt r}\,{\tt p}  *  \elhat b{\tt r}^+\,{\tt p}\,0}  *  \boxed[b]{\elhat a{\tt r}^*\,{\tt r}\,{\tt q}  *  \elhat a{\tt r}^+\,{\tt q}\,0}}}£
      £\ass{∃p'\,q'.\boxed[c]{\N a.\N b.\boxed[a]{\elhat b{\tt r}^*\,{\tt r}\,{\tt p} 
 *  \elhat b{\tt r}\,{\tt p}\,p'  *  \elhat b{\tt r}^+\,p'\,0} \\ 
 *  \boxed[b]{\elhat a{\tt r}^*\,{\tt r}\,{\tt q} 
 *  \elhat a{\tt r}\,{\tt q}\,q'  *  \elhat a{\tt r}^+\,q'\,0}}}£
      int* t = [p+1];
      int* u = [q+2];
      £\ass{\boxed[c]{\N a.\N b.\boxed[a]{\elhat b{\tt r}^*\,{\tt r}\,{\tt p} 
 *  \elhat b{\tt r}\,{\tt p}\,{\tt t}  *  \elhat b{\tt r}^+\,{\tt t}\,0} \\ 
 *  \boxed[b]{\elhat a{\tt r}^*\,{\tt r}\,{\tt q} 
 *  \elhat a{\tt r}\,{\tt q}\,{\tt u}  *  \elhat a{\tt r}^+\,{\tt u}\,0}}}£
      while (t!=0 && ...) do { p:=t; t:=[p+1]; }
      while (u!=0 && ...) do { q:=u; u:=[q+2]; }
      £\ass{\boxed[c]{\N a.\N b.\boxed[a]{\elhat b{\tt r}^*\,{\tt r}\,{\tt p} 
 *  \elhat b{\tt r}\,{\tt p}\,{\tt t}  *  \elhat b{\tt r}^+\,{\tt t}\,0} \\ 
 *  \boxed[b]{\elhat a{\tt r}^*\,{\tt r}\,{\tt q} 
 *  \elhat a{\tt r}\,{\tt q}\,{\tt u}  *  \elhat a{\tt r}^+\,{\tt u}\,0}}}£
    // end frame
    £\ass{\boxed[c]{\N a.\N b.\boxed[a]{\elhat b{\tt r}^*\,{\tt r}\,{\tt p} 
 *  \elhat b{\tt r}\,{\tt p}\,{\tt t}  *  \elhat b{\tt r}^+\,{\tt t}\,0} \\ 
 *  \boxed[b]{\elhat a{\tt r}^*\,{\tt r}\,{\tt q} 
 *  \elhat a{\tt r}\,{\tt q}\,{\tt u}  *  \elhat a{\tt r}^+\,{\tt u}\,0}} \\
 *  {\tt x}^1 \pto[.5] v  *  {\tt x}^2 ↦ \underscore  *  {\tt x}^3 ↦ \underscore}£
    [x+1]:=t;
    [x+2]:=u;
    £\ass{\boxed[c]{\N a.\N b.\boxed[a]{\elhat b{\tt r}^*\,{\tt r}\,{\tt p} 
 *  \elhat b{\tt r}\,{\tt p}\,{\tt t}  *  \elhat b{\tt r}^+\,{\tt t}\,0} \\ 
 *  \boxed[b]{\elhat a{\tt r}^*\,{\tt r}\,{\tt q} 
 *  \elhat a{\tt r}\,{\tt q}\,{\tt u}  *  \elhat a{\tt r}^+\,{\tt u}\,0}} \\ 
 *  {\tt x}^1 \pto[.5] v  *  {\tt x}^2 ↦ {\tt t}  *  {\tt x}^3 ↦ {\tt u}}£
    // tricky step
    £\ass{\boxed[c] {
el_0\,{\tt p}\,{\tt t}  *  el_1\,{\tt q}\,{\tt u}  *  \\ ∀n∈\{0,1\}.∀x.{}\\{}
\left(\ml[c]{( el_0\,{\tt p}\,x  *  el_0^n\,x\,{\tt t}  *  el_1\,{\tt q}\,x  *  el_1^n\,x\,{\tt u} )  \magicwand  \\
\N a.\N b.\boxed[a]{\elhat b{\tt r}^*\,{\tt r}\,{\tt p} 
 *  \elhat b{\tt r}\,{\tt p}\,x  *  \elhat b{\tt r}^n\,x\,{\tt t}  *  \elhat b{\tt r}^+\,{\tt t}\,0} \\ 
 *  \boxed[b]{\elhat a{\tt r}^*\,{\tt r}\,{\tt q} 
 *  \elhat a{\tt r}\,{\tt q}\,x  *  \elhat a{\tt r}^n\,x\,{\tt u}  *  \elhat a{\tt r}^+\,{\tt u}\,0}}\right)} \\
 *  {\tt x}^1 \pto[.5] v  *  {\tt x}^2 ↦ {\tt t}  *  {\tt x}^3 ↦ {\tt u}}£
    // begin action £$\Add{\tt r}\,{\tt x}$£ on region £$c$£
      £\ass{el_0\,{\tt p}\,{\tt t}  *  el_1\,{\tt q}\,{\tt u}  *  {\tt x}^1 \pto[.5] v * {\tt x}^2↦{\tt t} * {\tt x}^3↦{\tt u}}£
      [p+1]:=x;
      [q+2]:=x;
      £\ass{el_0\,{\tt p}\,{\tt x}  *  el_1\,{\tt q}\,{\tt x}  *  {\tt x}^1 \pto[.5] v * {\tt x}^2↦{\tt t} * {\tt x}^3↦{\tt u}}£
      £\ass{el_0\,{\tt p}\,{\tt x}  *  el_0\,{\tt x}\,{\tt t}  *  el_1\,{\tt q}\,{\tt x}  *  el_1\,{\tt x}\,{\tt u}}£
    // end action
    £\ass{\boxed[c] {(el_0\,{\tt p}\,{\tt x}  *  el_0\,{\tt x}\,{\tt t}  *  el_1\,{\tt q}\,{\tt x}  *  el_1\,{\tt x}\,{\tt u})  *  \\ ∀n∈\{0,1\}.∀x.{}\\{}
\left(\ml[c]{( el_0\,{\tt p}\,x  *  el_0^n\,x\,{\tt t}  *  el_1\,{\tt q}\,x  *  el_1^n\,x\,{\tt u} )  \magicwand  \\
\N a.\N b.\boxed[a]{\elhat b{\tt r}^*\,{\tt r}\,{\tt p} 
 *  \elhat b{\tt r}\,{\tt p}\,x  *  \elhat b{\tt r}^n\,x\,{\tt t}  *  \elhat b{\tt r}^+\,{\tt t}\,0} \\ 
 *  \boxed[b]{\elhat a{\tt r}^*\,{\tt r}\,{\tt q} 
 *  \elhat a{\tt r}\,{\tt q}\,x  *  \elhat a{\tt r}^n\,x\,{\tt u}  *  \elhat a{\tt r}^+\,{\tt u}\,0}}\right)}}£
    £\ass{\boxed[c] {\N a.\N b.\boxed[a]{\elhat b{\tt r}^*\,{\tt r}\,{\tt p} 
 *  \elhat b{\tt r}\,{\tt p}\,{\tt x}  *  \elhat b{\tt r}\,{\tt x}\,{\tt t}  *  \elhat b{\tt r}^+\,{\tt t}\,0} \\ 
 *  \boxed[b]{\elhat a{\tt r}^*\,{\tt r}\,{\tt q} 
 *  \elhat a{\tt r}\,{\tt q}\,{\tt x}  *  \elhat a{\tt r}^n\,{\tt x}\,{\tt u}  *  \elhat a{\tt r}^+\,{\tt u}\,0}}}£
    £\ass{\boxed[c]{{\tt x}∈\emph{2list}\,{\tt r}}}£
  // end frame
}
£\ass{\boxed[c]{{\tt x}∈\emph{2list}\,{\tt r}}  *  {\tt x}^1\pto[.5]v}£

\end{lstlisting}





\end{document}