\documentclass[12pt,a4paper]{article}
\usepackage{a4wide}
\usepackage{JohnMath}

\usepackage[svgnames]{xcolor}

% CODE LISTINGS
\usepackage{listings}
\lstset{
  language=C,
  columns=[l]fullflexible,
  mathescape=true,
  basicstyle=\ttfamily\color{Purple},
  showstringspaces=false,
  commentstyle=\color{DarkGreen}, 
  numbers=none, 
  escapechar=£,
  escapebegin=\normalsize\rmfamily\color{Black}
}

% SPECIFICATIONS
\newcommand{\ml}[2][t]{\mbox{\mdseries\begin{tabular}[#1]{@{}L@{}}#2\end{tabular}}}
\newcommand{\ass}[1]{\ensuremath{{\color{blue}\left\{\ml[c]{#1}\right\}}}}
\newcommand{\seqspec}[3]{\ass{#1}\,{\mbox{{\tt #2}}}\,\ass{#3}}
\newcommand{\Seqspec}[3]{\multicolumn{2}{l}{$\ass{#1}$ {#2} $\ass{#3}$}}
\newcommand{\comm}[1]{\vspace{-2pt}%
    \begin{list}{/$*$}{%
        \topsep=5pt%
        \leftmargin=3cm%
      }\item #1 \hfill$*$/\end{list}%
}

\renewcommand{\arraystretch}{1.2}

\title{Specifying overlaid datastructures}
\author{John Wickerson}
\date{}

\begin{document}

\maketitle

\subsection*{Preliminaries}

Suppose $R$ and $S$ are of type $\emph{expr} → \emph{expr} → \emph{assertion}$. Define:
\[
\begin{array}{rcl}
R;S &=& λx\,z.∃y.R\,x\,y ∧ S\,y\,z \\
R ∨ S &=& λx\,y. R\,x\,y ∨ S\,x\,y \\
\id &=& λx\, y
.x=y ∧ \emp
\end{array}
\]

\noindent Then let $R^*$ be the least function satisfying $R^* = \id ∨ (R;R^*)$ and let $R^+ = R;R^*$.

\section{A singly-indexed list}

Our first datastructure is a singly-indexed list. Every node has a value and a pointer to the next node. The final node's next pointer is set to 0. The first node is a sentinel, at a fixed location $r$. Our datastructure can be described by the following formula:
\[
el^+\,r\,0
\]
where:
\[
el\,x\,y  =  x^1 \pto[.5] \underscore  *  x^2↦y
\]
Note that the value field is only a half-pointer, making it read-only. (The other half is retained by the client who added that element to the list.) The datastructure supports two operations: inserting an element, and removing an element. The implementations of these two methods are as follows:

\begin{lstlisting}
insert(x){
  int* p = r;
  while ([p+1]!=0 && ...) do p:=[p+1];
  [x+1]:=[p+1];
  [p+1]:=x;
}
\end{lstlisting}

\begin{lstlisting}
remove(x){
  int* p = r;
  while ([p+1]!=x) do p:=[p+1];
  [p+1]:=[x+1];
}
\end{lstlisting}

\noindent We can specify these two methods like so:
\[
\begin{array}{c}
\seqspec{\boxed{el^+\,r\,0} * x^1 ↦ v * x^2 ↦ \underscore}{insert(x)}{\boxed{el^+\,r\,x * el^+\,x\,0} * x\pto[.5] v} \\
\seqspec{\boxed{el^+\,r\,x * el^+\,x\,0} * x^1\pto[.5] v}{remove(x)}{\boxed{el^+\,r\,0} * x^1 ↦ v * x^2 ↦ \underscore}
\end{array}
\]
The internal state of the datastructure is written inside a box, whereas the state owned by the client is written outside the box. Upon inserting an element $x$, the client receives a token, which is implemented as ``$x\pto[.5] v$''. When invoking the {\tt remove} method, the client must provide that token, as proof that he was the client who inserted the element. By this system of tokens, a client who has inserted an element $x$ can assert that the datastructure's internal state contains that element -- even though other calls to {\tt insert} and {\tt remove} may have occurred in the interim.

To formalise this reasoning, we use RGSep actions. The actions that are permitted on the datastructure's internal state are the following:
\[
\begin{array}{rr}
\textsc{Add}(x): & \boxed{el^*\,r\,p * true}  ⊦ p^2 ↦ p'  \rightsquigarrow  p^2 ↦ x * x^1\pto[.5] v * x^2↦p' \\
\textsc{Rm}(x): & x^1\pto[.5] v   *   \boxed{el^*\,r\,p * true}  ⊦ p^2 ↦ x * x^1\pto[.5] v * x^2↦p'  \rightsquigarrow  p^2 ↦ p'
\end{array}
\]
The purpose of the ``$el^*\,r\,w$'' guards is to ensure that $w$ is actually an element in the list. Without it, {\sc Add}'s precondition would be satisfied by an arbitrary heap cell. We require the client invoking $\textsc{Rm}(x)$ to have the token $x^1\pto[.5] v$ in his local state.

\subsection{Verification of the {\tt insert} method}

\begin{lstlisting}
£\ass{\boxed{el^+\,r\,0} * x^1 ↦ v * x^2↦\underscore}£
insert(x){
  int* p = r;
  £\ass{\boxed{el^*\,r\,p * el^+\,p\,0} * x^1 ↦ v * x^2↦\underscore}£
  while ([p+1]!=0 && ...) do p:=[p+1];
  £\ass{\boxed{el^*\,r\,p * el^+\,p\,0} * x^1 ↦ v * x^2↦\underscore}£
  £\ass{\boxed{el^*\,r\,p * p^1 \pto[.5]\underscore * p^2↦p' * el^*\,p'\,0} * x^1 ↦ v * x^2↦\underscore}£
  [x+1]:=[p+1];
  £\ass{\boxed{el^*\,r\,p * p^1 \pto[.5]\underscore * p^2↦p' * el^*\,p'\,0} * x^1 ↦ v * x^2↦p'}£
  // begin action Add(x)
    £\ass{el^*\,r\,p * p^1 \pto[.5]\underscore * p^2↦p' * el^*\,p'\,0 * x^1 ↦ v * x^2↦p'}£
    [p+1]:=x;
    £\ass{el^*\,r\,p * p^1 \pto[.5]\underscore * p^2↦x * el^*\,p'\,0  * x^1 ↦ v * x^2↦p'}£
  // end action
  £\ass{\boxed{el^*\,r\,p * p^1 \pto[.5]\underscore * p^2↦x * x^1\pto[.5]\underscore  *  x^2↦p' * el^*\,p'\,0} * x^1\pto[.5] v}£
}
£\ass{\boxed{el^+\,r\,x * el^+\,x\,0} * x^1\pto[.5] v}£
\end{lstlisting}

\subsection{Verifcation of the {\tt remove} method}

\begin{lstlisting}
£\ass{\boxed{el^+\,r\,x * el^+\,x\,0} * x^1\pto[.5] v}£
remove(x){
  int* p = r;
  £\ass{\boxed{el^*\,r\,p * el^+\,p\,x * el^+\,x\,0} * x^1\pto[.5] v}£
  while ([p+1]!=x) do p:=[p+1];
  £\ass{\boxed{el^*\,r\,p * p^1\pto[.5]\underscore * p^2↦x * el^+\,x\,0} * x^1\pto[.5] v}£
  £\ass{\boxed{el^*\,r\,p * p^1\pto[.5]\underscore * p^2↦x * x^1\pto[.5]\underscore * x^2↦y * el^*\,y\,0} * x^1\pto[.5] v}£
  // begin action Rm(x) 
    £\ass{el^*\,r\,p * p^1\pto[.5]\underscore * p^2↦x * el^*\,y\,0 * x^1↦v * x^2 ↦ y}£
    [p+1]:=[x+1];
    £\ass{el^*\,r\,p * p^1\pto[.5]\underscore * p^2↦y * el^*\,y\,0 * x^1↦v * x^2↦y}£
  // end action
  £\ass{\boxed{el^*\,r\,p * p^1\pto[.5]\underscore * p^2↦y * el^*\,y\,0} * x^1↦v * x^2↦y}£
}
£\ass{\boxed{el^+\,r\,0} * x^1 ↦ v * x^2↦\underscore}£
\end{lstlisting}

\newpage \section{A doubly-indexed list}

Let us move to a doubly-indexed list. Every node now has three fields: a value, and two next pointers. The two chains of next pointers present two orderings on the same set of nodes. Both orderings begin at the same sentinel node, which at a fixed location $r$. Our datastructure can be described by the following formula:
\[
el_0^+\,r\,0  ∧  el_1^+\,r\,0
\]
where:
\[
\begin{array}{rcl}
el_0\,x\,y &=& x^1 \pto[.5] \underscore  *  x^2↦y  *  x^3 ↦\underscore \\
el_1\,x\,y &=& x^1 \pto[.5] \underscore  *  x^2↦\underscore *  x^3 ↦y
\end{array}
\]

\noindent We can specify the insert method like so:
\begin{quote}
\begin{lstlisting}
£\ass{\boxed{el_0^+\,r\,0  ∧  el_1^+\,r\,0}  *  x^1 ↦ v  *  x^2 ↦ \underscore  *  x^3 ↦ \underscore}£
insert(x)
£\ass{\boxed{(el_0^+\,r\,x  *  el_0^+\,x\,0)  ∧  (el_1^+\,r\,x  *  el_1^+\,x\,0)}  *  x\pto[.5] v}£
\end{lstlisting}
\end{quote}

\noindent We can specify the remove method like so:
\begin{quote}
\begin{lstlisting}
£\ass{\boxed{(el_0^+\,r\,x  *  el_0^+\,x\,0) ∧ (el_1^+\,r\,x  *  el_1^+\,x\,0)}  *  x^1\pto[.5] v}£
remove(x)
£\ass{\boxed{el_0^+\,r\,0  ∧  el_1^+\,r\,0}  *  x^1 ↦ v  *  x^2 ↦ \underscore  *  x^3 ↦ \underscore}£
\end{lstlisting}
\end{quote}

\noindent The actions on the datastructure can be specified like so:
\[
\begin{array}{rl}
\textsc{Add}(x): & \boxed{(el_0^*\,r\,p  *  true)  ∧  (el_1^*\,r\,q  *  true)} \\
& ⊦ p^2 ↦ p'  *  q^3 ↦ q'   \rightsquigarrow  p^2 ↦ x *  q^3 ↦ x  * x^1\pto[.5] v * x^2↦p'  *  x^3 ↦ q' \\
\textsc{Rm}(x): & x^1\pto[.5] v   *   \boxed{(el_0^*\,r\,p  *  true)  ∧  (el_1^*\,r\,q  *  true)} \\
& ⊦ p^2 ↦ x *  q^3 ↦ x  * x^1\pto[.5] v * x^2↦p'  *  x^3 ↦ q'  \rightsquigarrow  p^2 ↦ p'  *  q^3 ↦ q'
\end{array}
\]

\noindent Note, in particular, that we cannot write {\sc Add}'s guard as 
\[
\boxed{(el_0^*\,r\,p  ∧  el_1^*\,r\,q)  *  true}
\]
because we don't know that the first ordering from $r$ to $p$ describes the same nodes as the second ordering from $r$ to $q$.

\subsection{Verifying the {\tt insert} method}

\begin{lstlisting}
£\ass{\boxed{el_0^+\,r\,0 ∧ el_1^+\,r\,0} * x^1 ↦ v * x^2↦\underscore * x^3↦\underscore}£
insert(x){
  int* p = r;
  int* q = r;
  £\ass{\boxed{(el_0^*\,r\,p * el_0^+\,p\,0) ∧ (el_1^*\,r\,q * el_1^+\,q\,0)} * x^1 ↦ v * x^2↦\underscore * x^3↦\underscore}£
  while ([p+1]!=0 && ...) do p:=[p+1];
  while ([q+2]!=0 && ...) do q:=[q+2];
  £\ass{\boxed{(el_0^*\,r\,p * el_0^+\,p\,0) ∧ (el_1^*\,r\,q * el_1^+\,q\,0)} * x^1 ↦ v * x^2↦\underscore * x^3↦\underscore}£
  £\ass{\boxed{(el_0^*\,r\,p * p^1\pto[.5]\underscore   *   p^2↦p' *  p^3↦\underscore  *  el_0^*\,p'\,0) ∧ {}\\{}
  (el_1^*\,r\,q * q^1\pto[.5]\underscore * q^2↦\underscore  *  q^3↦q'  *  el_1^*\,q'\,0)} {}\\{} 
  * x^1 ↦ v * x^2↦\underscore * x^3↦\underscore}£
  [x+1]:=[p+1];
  [x+2]:=[q+2];
  £\ass{\boxed{(el_0^*\,r\,p * p^1\pto[.5]\underscore   *   p^2↦p' *  p^3↦\underscore  *  el_0^*\,p'\,0) ∧ {}\\{}
  (el_1^*\,r\,q * q^1\pto[.5]\underscore * q^2↦\underscore  *  q^3↦q'  *  el_1^*\,q'\,0)} {}\\{}
  * x^1 ↦ v * x^2↦p' * x^3↦q'}£
  // begin action Add(x)
    £\ass{((el_0^*\,r\,p * p^1\pto[.5]\underscore   *   p^2↦p' *  p^3↦\underscore  *  el_0^*\,p'\,0) ∧ {}\\{}
  (el_1^*\,r\,q * q^1\pto[.5]\underscore * q^2↦\underscore  *  q^3↦q'  *  el_1^*\,q'\,0)) {}\\{}
  * x^1 ↦ v * x^2↦p' * x^3↦q'}£
    // note that £$(P∧Q)*R ⇔ (P*R) ∧ (Q*R)$£ when £$R$£ is precise 
    £\ass{((el_0^*\,r\,p * p^1\pto[.5]\underscore   *   p^2↦\underscore *  p^3↦\underscore  *  x^1 \pto[.5] \underscore * x^2↦p' * x^3↦\underscore   *  el_0^*\,p'\,0) ∧ {}\\{}
  (el_1^*\,r\,q * q^1\pto[.5]\underscore * q^2↦\underscore  *  q^3↦\underscore  * x^1 \pto[.5] \underscore * x^2↦\underscore * x^3↦q'  *  el_1^*\,q'\,0))  *  x^1 \pto[.5] v}£
    // begin frame
      £\ass{(el_0^*\,r\,p * p^1\pto[.5]\underscore   *   p^2↦\underscore *  p^3↦\underscore  *  x^1 \pto[.5] \underscore * x^2↦p' * x^3↦\underscore   *  el_0^*\,p'\,0) ∧ {}\\{}
  (el_1^*\,r\,q * q^1\pto[.5]\underscore * q^2↦\underscore  *  q^3↦\underscore  * x^1 \pto[.5] \underscore * x^2↦\underscore * x^3↦q'  *  el_1^*\,q'\,0)}£
      [p+1]:=x;
      [q+2]:=x;
      £\ass{(el_0^*\,r\,p * p^1\pto[.5]\underscore   *   p^2↦x *  p^3↦\underscore   *  x^1 \pto[.5] \underscore * x^2↦p' * x^3↦\underscore   *  el_0^*\,p'\,0) ∧ {}\\{}
  (el_1^*\,r\,q * q^1\pto[.5]\underscore * q^2↦\underscore  *  q^3↦x  * x^1 \pto[.5] \underscore * x^2↦\underscore * x^3↦q'  *  el_1^*\,q'\,0)}£
    // end frame
    £\ass{((el_0^*\,r\,p * p^1\pto[.5]\underscore   *   p^2↦x *  p^3↦\underscore  *  x^1 \pto[.5] \underscore * x^2↦p' * x^3↦\underscore   *  el_0^*\,p'\,0) ∧ {}\\{}
  (el_1^*\,r\,q * q^1\pto[.5]\underscore * q^2↦\underscore  *  q^3↦x  * x^1 \pto[.5] \underscore * x^2↦\underscore * x^3↦q'  *  el_1^*\,q'\,0))  *  x^1 \pto[.5] v}£
  // end action
  £\ass{\boxed{(el_0^*\,r\,p * p^1\pto[.5]\underscore   *   p^2↦x *  p^3↦\underscore   *  x^1 \pto[.5] \underscore * x^2↦p' * x^3↦\underscore   *  el_0^*\,p'\,0) ∧ {}\\{}
  (el_1^*\,r\,q * q^1\pto[.5]\underscore * q^2↦\underscore  *  q^3↦x   * x^1 \pto[.5] \underscore * x^2↦\underscore * x^3↦q'  *  el_1^*\,q'\,0 )} {}\\{}
  * x^1 \pto[.5] v}£
}
£\ass{\boxed{(el_0^+\,r\,x * el_0^+\,x\,0) ∧ (el_1^+\,r\,x * el_1^+\,x\,0)} * x^1\pto[.5] v}£
\end{lstlisting}

\noindent It remains to justify both the $\textsc{Add}(x)$ action, and the following step:
\begin{lstlisting}
£\ass{(el_0^*\,r\,p * p^1\pto[.5]\underscore   *   p^2↦\underscore *  p^3↦\underscore  *  x^1 \pto[.5] \underscore * x^2↦p' * x^3↦\underscore   *  el_0^*\,p'\,0) ∧ {}\\{}
  (el_1^*\,r\,q * q^1\pto[.5]\underscore * q^2↦\underscore  *  q^3↦\underscore  * x^1 \pto[.5] \underscore * x^2↦\underscore * x^3↦q'  *  el_1^*\,q'\,0)}£
[p+1]:=x;
[q+2]:=x;
£\ass{(el_0^*\,r\,p * p^1\pto[.5]\underscore   *   p^2↦x *  p^3↦\underscore   *  x^1 \pto[.5] \underscore * x^2↦p' * x^3↦\underscore   *  el_0^*\,p'\,0) ∧ {}\\{}
  (el_1^*\,r\,q * q^1\pto[.5]\underscore * q^2↦\underscore  *  q^3↦x  * x^1 \pto[.5] \underscore * x^2↦\underscore * x^3↦q'  *  el_1^*\,q'\,0)}£
\end{lstlisting}

\noindent We propose to use the following conjunction rule:
\[
\infer{\seqspec{P_1}C{Q_1} \\ \seqspec{P_2}C{Q_2}}{\seqspec{P_1 ∧ P_2}C{Q_1 ∧ Q_2}}
\]

\noindent The first required antecedant is the following (the other will follow by symmetry):
\begin{lstlisting}
£\ass{el_0^*\,r\,p * p^1\pto[.5]\underscore   *   p^2↦\underscore *  p^3↦\underscore  *  x^1 \pto[.5] \underscore * x^2↦p' * x^3↦\underscore   *  el_0^*\,p'\,0}£
[p+1]:=x;
[q+2]:=x;
£\ass{el_0^*\,r\,p * p^1\pto[.5]\underscore   *   p^2↦x *  p^3↦\underscore   *  x^1 \pto[.5] \underscore * x^2↦p' * x^3↦\underscore   *  el_0^*\,p'\,0}£
\end{lstlisting}

\noindent This reduces to:
\begin{lstlisting}
£\ass{el_0^*\,r\,p * p^1\pto[.5]\underscore   *   p^2↦x *  p^3↦\underscore   *  x^1 \pto[.5] \underscore * x^2↦p' * x^3↦\underscore   *  el_0^*\,p'\,0}£
[q+2]:=x;
£\ass{el_0^*\,r\,p * p^1\pto[.5]\underscore   *   p^2↦x *  p^3↦\underscore   *  x^1 \pto[.5] \underscore * x^2↦p' * x^3↦\underscore   *  el_0^*\,p'\,0}£
\end{lstlisting}

\noindent This is broadly similar to proving: 
\[
\seqspec{el_0^+\,r\,0}{[q+2]:=x}{el_0^+\,r\,0}
\]

\noindent Intuitively, this is sensible; the $el_0^+$ predicate is sensitive only to the values of the \emph{second} fields of nodes, so updating the \emph{third} field of node $q$ should preserve the predicate. But we are making assumptions about $q$; namely, that it corresponds to a valid node in the list. For if $q$ is an arbitrary heap address, then its third field, located at $q+2$, could coincide with the second field of some node in the list, and in this case, the postcondition would fail. Of course, we \emph{are} justified in making this assumption about $q$, because we know that $el_1^*\,r\,q$ holds. Unfortunately, we deleted this fact when we applied the conjunction rule. It's not clear how we could have kept that fact around.

We could avoid this hassle by assuming list nodes to be nicely aligned in memory -- i.e., that the address of the first cell of any node is divisible by 3. However, we don't actually want to make this restriction. Morally speaking, we should be able to do without it, and practically speaking, dlmalloc's structures are by no means `nicely aligned'.

The other difficulty remaining is to show that the routine provides a valid instance of the $\textsc{Add}(x)$ action. This is pretty much the same problem.

\section{Co-referring regions}

We propose to describe the datastructure in a very different way. We shall see it as two \emph{separate} lists (that is, we will use separating conjunction where previously we had ordinary conjunction). But in order to preserve the close relationship between the two lists (namely, that every node appearing in one list also appears in the other) we shall use `ghost pointers', which map each element of one list to its counterpart in the other list. Here is our first attempt (note that we amend the assertion language of RGSep so as to allow nested boxes):
\[
\boxed[0]{\widehat{el_0}^+(r,0)}  *  \boxed[1]{\widehat{el_1}^+(r,0)}
\]
where:
\[
\begin{array}{rcl}
el_0\,x\,y &\iffdef& x^1 \pto[.25] \underscore  *  x^2↦y \\
el_1\,x\,y &\iffdef& x^1 \pto[.25] \underscore  *  x^3↦y
\end{array}
\]
and:
\[
\begin{array}{rcl}
in_0\,x \iffdef \boxed[0]{el_0^*(r,x) * \true} \\
in_1\,x \iffdef \boxed[1]{el_1^*(r,x) * \true}
\end{array}
\]
and:
\[
\begin{array}{rcl}
\widehat{el_0}\,x\,y &\iffdef& el_0\,x\,y  *  in_1\,x \\
\widehat{el_1}\,x\,y &\iffdef& el_1\,x\,y  *  in_0\,x
\end{array}
\]

\noindent The predicate $\widehat{el_0}$ describes an element that appears in the first list. It uses the $in_1$ predicate to specify that the element appears in the second list too. From this and the symmetric fact about $\widehat{el_1}$, we can deduce that the two lists pass through exactly the same set of elements, as required.

We now specify the insert method like so:

\begin{quote}
\begin{lstlisting}
£\ass{\boxed[0]{\widehat{el_0}^+\,r\,0}  *  \boxed[1]{\widehat{el_1}^+\,r\,0}  *  x^1 ↦ v  *  x^2 ↦ \underscore  *  x^3 ↦ \underscore}£
insert(x)
£\ass{\boxed[0]{\widehat{el_0}^+\,r\,x  *  \widehat{el_0}^+\,x\,0}  *  \boxed[1]{\widehat{el_1}^+\,r\,x  *  \widehat{el_1}^+\,x\,0}  *  x\pto[.5] v}£
\end{lstlisting}
\end{quote}

\noindent We can specify the remove method like so:
\begin{quote}
\begin{lstlisting}
£\ass{\boxed[0]{\widehat{el_0}^+\,r\,x  *  \widehat{el_0}^+\,x\,0}  *  \boxed[1]{\widehat{el_1}^+\,r\,x  *  \widehat{el_1}^+\,x\,0}  *  x\pto[.5] v}£
remove(x)
£\ass{\boxed[0]{\widehat{el_0}^+\,r\,0}  *  \boxed[1]{\widehat{el_1}^+\,r\,0}  *  x^1 ↦ v  *  x^2 ↦ \underscore  *  x^3 ↦ \underscore}£
\end{lstlisting}
\end{quote}

\noindent The actions on the datastructure can be specified like so:
\[
\begin{array}{rl}
\textsc{Add}(x): & \boxed[0]{el_0^*\,r\,p  *  true}  *  \boxed[1]{el_1^*\,r\,q  *  true} \\
& ⊦ \left(\begin{array}{l} 0  :  p^2 ↦ p'   \rightsquigarrow   p^2 ↦ x * x^1\pto[.25] v * x^2↦p' {}\\{} 1  :  q^3 ↦ q'    \rightsquigarrow   q^3 ↦ x  * x^1\pto[.25] v  *  x^3 ↦ q' \end{array}\right)\\[15pt]
\textsc{Rm}(x): & x^1\pto[.5] v   *   \boxed[0]{el_0^*\,r\,p  *  true}  *  \boxed[1]{el_1^*\,r\,q  *  true} \\
& ⊦ \left(\begin{array}{l} 0  :  p^2 ↦ x * x^1\pto[.25] v * x^2↦p'   \rightsquigarrow   p^2 ↦ p' {}\\{} 1  :  q^3 ↦ x  * x^1\pto[.25] v  *  x^3 ↦ q'    \rightsquigarrow   q^3 ↦ q' \end{array}\right)
\end{array}
\]

\noindent Here, the actions are updating multiple shared regions at the same time. The details of how this works have to be worked out. Now let us seek to verify the implementation of the insert method.

\subsection{Verifying the {\tt insert} method}

\begin{lstlisting}
£\ass{\boxed[0]{\widehat{el_0}^+\,r\,0} * \boxed[1]{\widehat{el_1}^+\,r\,0} * x^1 ↦ v * x^2↦\underscore * x^3↦\underscore}£
insert(x){
  int* p = r;
  int* q = r;
  £\ass{\boxed[0]{\widehat{el_0}^*\,r\,p * \widehat{el_0}^+\,p\,0} * \boxed[1]{\widehat{el_1}^*\,r\,q * \widehat{el_1}^+\,q\,0} * x^1 ↦ v * x^2↦\underscore * x^3↦\underscore}£
  while ([p+1]!=0 && ...) do p:=[p+1];
  while ([q+2]!=0 && ...) do q:=[q+2];
  £\ass{\boxed[0]{\widehat{el_0}^*\,r\,p * \widehat{el_0}^+\,p\,0} * \boxed[1]{\widehat{el_1}^*\,r\,q * \widehat{el_1}^+\,q\,0} * x^1 ↦ v * x^2↦\underscore * x^3↦\underscore}£
  £\ass{\boxed[0]{\widehat{el_0}^*\,r\,p * p^1\pto[.25]\underscore   *   p^2↦p'  *  in_1\,p  *  \widehat{el_0}^*\,p'\,0} * {}\\{}
  \boxed[1]{\widehat{el_1}^*\,r\,q * q^1\pto[.25]\underscore  *  q^3↦q'  *  in_0\,q  *  \widehat{el_1}^*\,q'\,0} {}\\{} 
  * x^1 ↦ v * x^2↦\underscore * x^3↦\underscore}£
  [x+1]:=[p+1];
  [x+2]:=[q+2];
  £\ass{\boxed[0]{\widehat{el_0}^*\,r\,p * p^1\pto[.25]\underscore   *   p^2↦p'  *  in_1\,p  *  \widehat{el_0}^*\,p'\,0} * {}\\{}
  \boxed[1]{\widehat{el_1}^*\,r\,q * q^1\pto[.25]\underscore  *  q^3↦q'  *  in_0\,q  *  \widehat{el_1}^*\,q'\,0} {}\\{} 
  * x^1 ↦ v * x^2↦p' * x^3↦q'}£
  // begin action Add(x)
    £\ass{\boxed[0]{\widehat{el_0}^*\,r\,p  *  \widehat{el_0}^*\,p'\,0} * 
  \boxed[1]{\widehat{el_1}^*\,r\,q  *  \widehat{el_1}^*\,q'\,0} {}\\{} 
 * p^1\pto[.25]\underscore   *   p^2↦p'  *  in_1\,p * q^1\pto[.25]\underscore  *  q^3↦q'  *  in_0\,q {}\\{} * x^1 ↦ v * x^2↦p' * x^3↦q'}£
    // begin frame
      £\ass{p^2↦p'  *  q^3↦q'}£
      [p+1]:=x;
      [q+2]:=x;
      £\ass{p^2↦x  *  q^3↦x}£
    // end frame
    £\ass{\boxed[0]{\widehat{el_0}^*\,r\,p  *  \widehat{el_0}^*\,p'\,0}  *  \boxed[1]{\widehat{el_1}^*\,r\,q  *  \widehat{el_1}^*\,q'\,0} {}\\{} 
    * p^1\pto[.25]\underscore   *   p^2↦x  *  in_1\,p  * q^1\pto[.25]\underscore  *  q^3↦x  *  in_0\,q {}\\{} 
    *  x^1↦\underscore  *  x^2 ↦ p'  *  x^3↦q'}£
  // end action
  £\ass{\boxed[0]{\widehat{el_0}^*\,r\,p * p^1\pto[.25]\underscore   *   p^2↦x  *  in_1\,p  *  x^1\pto[.25]\underscore  *  x^2 ↦ p' *  \widehat{el_0}^*\,p'\,0}  * {}\\{}
  \boxed[1]{\widehat{el_1}^*\,r\,q * q^1\pto[.25]\underscore  *  q^3↦x  *  in_0\,q  *  x^1\pto[.25]\underscore  * x^3↦q'  *  \widehat{el_1}^*\,q'\,0} {}\\{} 
  * x^1 \pto[.5] v}£
  // combine list segments
  £\ass{\boxed[0]{\widehat{el_0}^+\,r\,x  *  x^1\pto[.25]\underscore  *  x^2 ↦ p' *  \widehat{el_0}^*\,p'\,0}  * {}\\{}
  \boxed[1]{\widehat{el_1}^+\,r\,x  *  x^1\pto[.25]\underscore  * x^3↦q'  *  \widehat{el_1}^*\,q'\,0}  * x^1 \pto[.5] v}£
  // deduce £$in_1\,x$£ and £$in_0\,x$£
  £\ass{\boxed[0]{\widehat{el_0}^*\,r\,x  *  x^1\pto[.25]\underscore  *  x^2 ↦ p'  *  in_1\,x  *  \widehat{el_0}^*\,p'\,0}  *  {}\\{}
  \boxed[1]{\widehat{el_1}^*\,r\,x  *  x^1\pto[.25]\underscore  * x^3↦q'  *  in_0\,x  *  \widehat{el_1}^*\,q'\,0}  * x^1 \pto[.5] v}£
  // combine list segments
  £\ass{\boxed[0]{\widehat{el_0}^+\,r\,x * \widehat{el_0}^+\,x\,0}  *  \boxed[1]{\widehat{el_1}^+\,r\,x * \widehat{el_1}^+\,x\,0} * x^1\pto[.5] v}£
}
£\ass{\boxed[0]{\widehat{el_0}^+\,r\,x * \widehat{el_0}^+\,x\,0}  *  \boxed[1]{\widehat{el_1}^+\,r\,x * \widehat{el_1}^+\,x\,0} * x^1\pto[.5] v}£
\end{lstlisting}



\end{document}