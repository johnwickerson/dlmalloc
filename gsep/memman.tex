\documentclass[12pt,a4paper]{article}
\usepackage{a4wide}
\usepackage{JohnMath}
\usepackage{mathtools}

\usepackage[svgnames]{xcolor}

% CODE LISTINGS
\usepackage{listings}
\lstset{
  language=C,
  columns=[l]fullflexible,
  mathescape=true,
  basicstyle=\ttfamily\color{Black},
  showstringspaces=false,
  commentstyle=\color{DarkGreen}, 
  numbers=none, 
  escapechar=£,
  escapebegin={\normalsize\rmfamily\color{Black}}
}

% SPECIFICATIONS
\newcommand{\ml}[2][t]{\mbox{\mdseries\begin{tabular}[#1]{@{}L@{}}#2\end{tabular}}}
\newcommand{\ass}[1]{\ensuremath{{\color{blue}\left\{\ml[c]{#1}\right\}}}}
\newcommand{\seqspec}[3]{\ass{#1}\,{\mbox{{\tt #2}}}\,\ass{#3}}
\newcommand{\Seqspec}[3]{\multicolumn{2}{l}{$\ass{#1}$ {#2} $\ass{#3}$}}
\newcommand{\comm}[1]{\vspace{-2pt}%
    \begin{list}{/$*$}{%
        \topsep=5pt%
        \leftmargin=3cm%
      }\item #1 \hfill$*$/\end{list}%
}

\renewcommand{\arraystretch}{1.2}

\renewcommand{\true}{\mathsf{tt}}
\renewcommand{\emp}{\emph{emp}}

%\renewcommand{\boxed}[2][]{{\textbf{[}}#2{\textbf{]}}_{#1}}

\newcommand{\rsem}[1]{{(\![}{#1}{]\!)}}
\newcommand{\iterstar}[2][]{\text{\LARGE $*$}^{#1}_{#2}}

\newcommand{\defined}{\mathop{\text{def}}}

\newcommand{\SET}[2]{\left\{\begin{array}{@{}l|l@{}} #1 & #2 \end{array}\right\}}

\newenvironment{mapping}{\left\{ \begin{array}{@{}r@{\,↦\,}l@{}}}{\end{array}\right\}}

%\newcommand{\block}[1]{\smash{\overbracket[0.5pt][2pt]{\underscore\,\ldots\,\underscore}^{#1}}}

\newcommand{\ret}{\texttt{ret}}

\newcommand{\malloc}{{\tt malloc}}
\newcommand{\sbrk}{{\tt sbrk}}
\newcommand{\free}{{\tt free}}
\newcommand{\ls}[2][]{#1 \twoheadrightarrow #2}
\newcommand{\arena}{\mathop{\emph{arena}}}
\newcommand{\anArena}{\emph{anArena}}
\newcommand{\token}{\mathop{\emph{token}}}
\newcommand{\arenatoken}{\mathop{\emph{arenatoken}}}
\newcommand{\block}{\mathop{\emph{block}}}
\newcommand{\ublock}{\mathop{\emph{ublock}}}
\newcommand{\ablock}{\mathop{\emph{ablock}}}
\newcommand{\sblock}{\mathop{\emph{sblock}}}
\newcommand{\A}[2][]{#1 \mathbin{\smash{\underset{\raisebox{3.5pt}{\smash{\sf\scriptsize a}}}{\rightarrow}}} #2}
\newcommand{\U}[2][]{#1 \mathbin{\smash{\underset{\raisebox{3.5pt}{\smash{\sf\scriptsize u}}}{\rightarrow}}} #2}
\newcommand{\B}[2][]{#1 {\rightarrow} #2}
\newcommand{\E}[2]{#2 \mathbin{\raisebox{2pt}{$\curvearrowleft$}} #1}
\newcommand{\s}{{\tt s}}
\renewcommand{\t}{{\tt t}}
\newcommand{\temp}{{\tt temp}}
\newcommand{\p}{{\tt p}}
\newcommand{\q}{{\tt q}}
\renewcommand{\v}{{\tt v}}
\newcommand{\brk}{\mathop{\emph{brk}}}
\newcommand{\brka}{\mathop{\emph{brka}}}
\newcommand{\n}{{}\\{} \hfill }
\newcommand{\nw}{{\tt nw}}
\newcommand{\ap}{{\tt ap}}
\newcommand{\nbytes}{{\tt nbytes}}
\newcommand{\WORD}{{\tt WORD}}
\newcommand{\mathceil}[1]{\left\lceil#1\right\rceil}
\newcommand{\ptoU}{\mathbin{↦_{\sf u}}}
\newcommand{\ptoA}{\mathbin{↦_{\sf a}}}
\newcommand{\ptoS}{\mathbin{↦_{\sf s}}}
\newcommand{\uninit}{\mathop{\emph{uninit}}}
\newcommand{\arity}{\mathop{\mathrm{arity}}}

\title{Verifying Memory Managers using GSep}
\author{John Wickerson}

\begin{document}

\maketitle

\subsection*{Preliminary: Spatial closure operators}

Suppose $R$ and $S$ are of type $\emph{loc} → \emph{loc} → \emph{assertion}$. Define:
\[
\begin{array}{rcl}
R;S &\eqdef& λx\,z.∃y.R\,x\,y  *  S\,y\,z \\
R ∨ S &\eqdef& λx\,y. R\,x\,y ∨ S\,x\,y \\
\id &\eqdef& λx\, y.x=y ∧ \emp \\
R^* &\eqdef& μS.S = \id ∨ R;S \\
R^+ &\eqdef& R;R^*
\end{array}
\]

\noindent Then the ordinary \emph{list} predicate can be defined like so:
\[
\emph{list}(x) \eqdef (λx\,y.x↦y)^*\,x\,0
\] 

\noindent Furthermore, we can parameterise the definitions by an element $m$ of a partial commutative monoid (PCM) $(M,·,u)$. Define:
\[
\begin{array}{rcl}
R;S &\eqdef& λx\,z\,m.∃y\,m_1\,m_2.m=m_1·m_2  *  R\,x\,y\,m_1  *  S\,y\,z\,m_2 \\
R ∨ S &\eqdef& λx\,y\,m. R\,x\,y\,m ∨ S\,x\,y\,m \\
\id &\eqdef& λx\,y\,m.x=y ∧ m=u ∧ \emp  \\
R^* &\eqdef& μS.S = \id ∨ R;S \\
R^+ &\eqdef& R;R^*
\end{array}
\]

\noindent Firstly, using the PCM of sets of naturals, $(\pow{\mathbb{N}}, \uplus, ∅)$, we can define Bornat-style lists, which are parameterised by the set $X$ of locations through which they pass, like so:
\[
\emph{blist}(x,X) \eqdef (λx\,y\,X.x↦y ∧ X=\{x\})^*\,x\,0\,X
\]

\noindent Secondly, using the unique 1-element PCM, $(\{u\}, λ\underscore\,\underscore.u, u)$, the extra parameters become redundant, and can be removed in such a way as to restore the original version above.

Thirdly, we can define an arena that comprises a chain of unallocated, allocated, and system blocks -- more on this later.

\begin{lemma} $R^* = (R^*;R^*)$.
\end{lemma}

\section{A fixed-sized allocator}

\subsubsection*{External spec}
\[
\begin{array}{c}
⊦\seqspec{\emp}{malloc()}{\ret↦\underscore\,\underscore} \\
⊦\seqspec{{\tt x}↦\underscore\,\underscore}{free(x)}{\emp}
\end{array}
\]

\subsection{Implementation using a free list}

\subsubsection*{Internal spec}

The external spec can be derived from the following internal spec using the hypothetical frame rule.

\[
\begin{array}{c}
Δ ⊦ \seqspec{\emph{list}\,{\tt f}}{malloc()}{\ret↦\underscore\,\underscore * \emph{list}\,{\tt f}} \\
Δ ⊦ \seqspec{{\tt x}↦\underscore\,\underscore * \emph{list}\,{\tt f}}{free(x)}{\emph{list}\,{\tt f}}
\end{array}
\]

\noindent where $Δ$ defines:
\[
\emph{list}\,x   \eqdef   x=0 ∧ \emp  ∨  ∃y.x ↦ y\,\underscore * \emph{list}\,y
\]

\subsubsection*{Verification}

\begin{lstlisting}
union list {
  union list* ptr;
  long x;
};

static union list f = NULL; £\ass{\emph{list}\,{\tt f}}£

void* malloc() 
£\ass{\emph{list}\,{\tt f}}£ 
{
  £\ass{{\tt f}=0 ∧ \emp  ∨  ∃y.{\tt f} ↦ y\,\underscore * \emph{list}\,y}£
  void* x;
  if (f == NULL) 
    £\ass{{\tt f}=0 ∧ \emp}£
    x = cons(2);
    £\ass{{\tt x}↦\underscore\,\underscore * ({\tt f}=0 ∧ \emp)}£
    £\ass{{\tt x}↦\underscore\,\underscore * \emph{list}\,{\tt f}}£
  else {
    £\ass{∃y.{\tt f} ↦ y\,\underscore * \emph{list}\,y}£
    x = f; 
    £\ass{∃y.{\tt x} ↦ y\,\underscore * \emph{list}\,y}£
    f = x->ptr;
    £\ass{{\tt x} ↦ {\tt f}\,\underscore * \emph{list}\,{\tt f}}£
    £\ass{{\tt x} ↦ \underscore\,\underscore * \emph{list}\,{\tt f}}£
  }
  £\ass{{\tt x} ↦ \underscore\,\underscore * \emph{list}\,{\tt f}}£
  return (void *)x;
  £\ass{\false}£
}
£\ass{\ret↦\underscore\,\underscore * \emph{list}\,{\tt f}}£


void free(void* ax)
£\ass{{\tt ax}↦\underscore\,\underscore * \emph{list}\,{\tt f}}£
{
  List* x = (List*)ax;
  £\ass{{\tt x}↦\underscore\,\underscore * \emph{list}\,{\tt f}}£
  x -> ptr = f;
  £\ass{{\tt x}↦{\tt f}\,\underscore * \emph{list}\,{\tt f}}£
  £\ass{∃y.{\tt x}↦y\,\underscore * \emph{list}\,y}£
  £\ass{\emph{list}\,{\tt x}}£
  f = x;  
} 
£\ass{\emph{list}\,{\tt f}}£
\end{lstlisting}


\section{A variable-sized allocator}

\subsubsection*{External spec}
\[
\begin{array}{c}
⊦ \seqspec{\emp}{malloc(n)}{(\emph{token}\,\ret\,\mathceil{{\tt n}/\WORD}  *  \iterstar[\mathceil{{\tt n}/\WORD}-1]{i=0}.\ret+i↦\underscore) ∨ \ret=0} \\
⊦ \seqspec{∃n.\emph{token}\,{\tt x}\,n  *  \iterstar[n-1]{i=0}.{\tt x}+i↦\underscore}{free(x)}{\emp}
\end{array}
\]


\subsection{Na\"ive implementation (no free list)}

\subsubsection*{Internal spec}
The external spec can be derived from the following internal spec using the rule for weakening predicate environments.

\[
\begin{array}{c}
Δ ⊦ \seqspec{\emp}{x := malloc(n)}{\emph{token}\,\ret\,{\tt n}  *  \iterstar[{\tt n}-1]{i=0}.\ret+i↦\underscore} \\
Δ ⊦ \seqspec{∃n.\emph{token}\,{\tt x}\,n  *  \iterstar[n-1]{i=0}.{\tt x}+i↦\underscore}{free(x)}{\emp}
\end{array}
\]
\noindent where $Δ$ defines:
\[
\begin{array}{rcl}
\emph{token}\,x\,n &\eqdef& (x-1)↦n
\end{array}
\]

\subsubsection*{Verification}

\begin{lstlisting}
void* malloc(int n) 
£\ass{\emp}£
{
  void* x = cons(n+1);
  £\ass{\iterstar[{\tt n}]{i=0}.({\tt x}+i)↦\underscore}£
  *x = n;
  £\ass{{\tt x}↦{\tt n}  *  \iterstar[{\tt n}]{i=1}.({\tt x}+i)↦\underscore}£
  £\ass{{\tt x}↦{\tt n}  *  \iterstar[{\tt n}-1]{i=0}.({\tt x}+1+i)↦\underscore}£
  £\ass{\emph{token}\,({\tt x}+1)\,{\tt n}  *  \iterstar[{\tt n}-1]{i=0}.({\tt x}+1+i)↦\underscore}£
  return x+1;
  £\ass{\false}£
}
£\ass{\emph{token}\,\ret\,{\tt n}  *  \iterstar[{\tt n}-1]{i=0}.(\emph{ret}+i)↦\underscore}£

void free(void* x) 
£\ass{∃n.\emph{token}\,{\tt x}\,n  *  \iterstar[n-1]{i=0}.({\tt x}+i)↦\underscore}£
{
  £\ass{∃n.({\tt x}-1)↦n  *  \iterstar[n-1]{i=0}.({\tt x}+i)↦\underscore}£
  int n = *(x-1);
  £\ass{({\tt x}-1)↦{\tt n}  *  \iterstar[{\tt n}-1]{i=0}.({\tt x}+i)↦\underscore}£
  £\ass{\iterstar[{\tt n}-1]{i=-1}.({\tt x}+i)↦\underscore}£
  £\ass{\iterstar[{\tt n}-1]{i=-1}.({\tt x}+i)↦\underscore}£
  while (n>=0) {
    £\ass{{\tt n}≥0 ∧ \iterstar[{\tt n}-1]{i=-1}.({\tt x}+i)↦\underscore}£
    n--;
    £\ass{({\tt x}+{\tt n})↦\underscore  *  \iterstar[{\tt n}-1]{i=-1}.({\tt x}+i)↦\underscore}£
    dispose(x+n);
    £\ass{\iterstar[{\tt n}-1]{i=-1}.({\tt x}+i)↦\underscore}£
  }
  £\ass{{\tt n}<0 ∧ \iterstar[{\tt n}-1]{i=-1}.({\tt x}+i)↦\underscore}£
}
£\ass{\emp}£
\end{lstlisting}

\subsection{Second implementation (Unix V7)}

Note that the various `pure' operators, such as `$=$' and `$>$' and `$\defined(-)$', are all given an empty footprint. That is, read $x=5$ as $x=5 ∧ \emp$.

The external spec can be derived from the following internal spec using the hypothetical frame rule (which removes the invariant $\anArena$), the rule for weakening predicate environments (which removes $Δ$), and the {\sc Erase} rule (which removes $G$).

\subsubsection*{Internal spec}

\[
\begin{array}{c}
δ;γ;G ⊦ \seqspec{\anArena}{malloc(n)}{\anArena  *  {}\\{} ((\token\ret\,\mathceil{{\tt n}/\WORD}  *  \iterstar[\mathceil{{\tt n}/\WORD}-1]{i=0}.\ret+i↦\underscore) ∨ \ret = 0 ) } \\
δ;γ;G ⊦ \seqspec{\anArena  *  ∃n.\token\,{\tt x}\,n  *  \iterstar[n-1]{i=0}.{\tt x}+i↦\underscore}{free(x)}{\anArena}
\end{array}
\]
\noindent where $δ$ defines:
\[
\begin{array}{rcl}
\ublock x\,y\,B &\eqdef& B = \{x+1 \ptoU y-x-1\}  *  x<y  *  x↦y  *  \iterstar[y-1]{i=x+1}.i↦\underscore \\
\ablock x\,y\,B &\eqdef& B = \{x+1 \ptoA y-x-1\}  *  x<y  *  x_{\mid 1}\pto[.5]y \\
\sblock x\,y\,B &\eqdef& B = \{x+1 \ptoS y-x-1\}  *  x<y  *  x_{\mid 1}↦y \\
\block &\eqdef& {\ublock} ∨ {\ablock} ∨ {\sblock} \\
\uninit\,A &\eqdef& \s↦0\,0  *  A=∅  *  \brka(\s+2) \\
\arena\,A &\eqdef& \ml{∃B_1,B_2 : \mathcal{B}.\block^*\,\s\,\v\,B_1  *  \block^*\v\,\t\,B_2 {}\\{} *  A = (B_1 \uplus B_2)^{\sf a}  *  \t_{\mid1} ↦ \s  *  \brka(\t+1)} \\
\anArena &\eqdef& \boxed{∃A.\uninit\,A  ∨  \arena\,A} \\
\token x\,n &\eqdef& \boxed{∃A.\arena(A \uplus \{x↦n\})}  *  (x-1)_{\mid 1}\pto[.5] x+n 
\end{array}
\]
\noindent Note that we use the following separation algebra for the spatial closure operators:
\[
\mathcal{B} \eqdef (\mathbb{N}\rightharpoonup \{{\sf u},{\sf a},{\sf s}\}×\mathbb{N}_0, \uplus, ∅)
\]

\noindent Note also that $B^{\sf a}$ returns a function of type $\mathbb{N}\rightharpoonup\mathbb{N}_0$, such that $(x↦n) ∈ B^{\sf a}$ if and only if $(x\ptoA n) ∈ B$. 

The guarantee $G$ is defined as $\bigcup_x\,\left\{\emph{Malloc}, \emph{Free}\,x\right\}$, where:
\[
\begin{array}{rcl}
\emph{Malloc} &\eqdef& ∃A,x,n.(\s↦0\,0  *  A=∅)  ∨  arena\,A  \rightsquigarrow \arena(A\uplus\{x↦n\}) \\
\emph{Free}\,x &\eqdef& ∃A,n.(x-1)_{\mid 1}\pto[.5](x+n)  \mid  \arena(A\uplus\{x↦n\})  \rightsquigarrow \arena A \\
\end{array}
\]
\noindent The procedure environment $γ$ provides a specification for \sbrk. The `official' spec for \sbrk is as follows:
\[
⊦ \seqspec{\brk(b)}{sbrk(n)}{(\brk(b)  *  \ret=-1  *  {\tt n}≠0) ∨ {}\\{} (\brk(b+\mathceil{{\tt n}/\WORD})  *  \ret=b  *  \iterstar[\mathceil{{\tt n}/\WORD}-1]{i=0}.\ret+i↦\underscore)} \\
\]

\noindent but if we define $\brka(x)$ as shorthand for $∃b≥x.\brk(b)$, then we obtain the following derived spec:
\[
⊦ \seqspec{\brka(x)}{sbrk(n)}{(\brka(x)  *  \ret=-1  *  {\tt n}≠0) ∨ {}\\{} (\brka(\ret+\mathceil{{\tt n}/\WORD})  *  x≤\ret  *  \iterstar[\mathceil{{\tt n}/\WORD}-1]{i=0}.\ret+i↦\underscore)} \\
\]

\noindent which is easier to use, and is hence the one contained in $γ$.

The verification of the module depends on the following two lemmas:

\begin{lemma}
\emph{$\block^*x_1\,y_1\,B_1  *  \block^*x_2\,y_2\,B_2  ⟹  B_1 \perp B_2$}
\end{lemma}
\begin{lemma}
\emph{$\block^*x\,y\,B  *  w↦z  ⟹  w+1 ∉ \dom(B)$}
\end{lemma}

\subsubsection*{Verification of malloc routine}

\begin{lstlisting}
#define WORD sizeof(union store)
#define BLOCK 1024	/* a multiple of WORD*/
#define testbusy(p) ((int)(p)&1)
#define setbusy(p) (struct store *)((int)(p)|1)
#define clearbusy(p) (struct store *)((int)(p)&~1)

struct store {struct store *ptr;};
static struct store s[2]; /* initial arena */
static struct store *v;   /* search ptr */
static struct store *t;   /* arena top */

char *malloc(unsigned int nbytes) 
£\ass{\anArena}£
£\ass{\boxed{∃A.\uninit A  ∨  \arena A}}£
// begin Existential
£\ass{\boxed{\uninit A  ∨  \arena A}}£
// begin region update (action is either Malloc or none)
£\ass{\uninit A  ∨  \arena A}£
// Precondition for returning:
£\ass{\left(\ml[c]{\arena(A\uplus\{\ret↦\mathceil{\nbytes/\WORD}\}) {}\\{}
*  \iterstar[\mathceil{\nbytes/\WORD}-1]{i=0}.\ret+i↦\underscore {}\\{}
*  (\ret-1)_{\mid 1}\pto[.5]\ret+\mathceil{\nbytes/\WORD}}\right) ∨ (\arena\,A  *  \ret=0)}£
{
  £\ass{\uninit A  ∨  \arena A}£
  register struct store *p, *q;
  register nw;
  static temp;
  if(s[0].ptr == 0) { /*first time*/
    £\ass{\uninit A}£
    £\ass{\s↦0\,0  *  \brka(\s+2)  *  A=∅}£
    s[0].ptr = setbusy(&s[1]);
    £\ass{\s_{\mid 1}↦\s+1  *  \s+1↦0  *  \brka(\s+2)  *  A=∅}£
    s[1].ptr = setbusy(&s[0]);
    £\ass{\s_{\mid 1}↦\s+1  *  (\s+1)_{\mid 1}↦\s  *  \brka(\s+2)  *  A=∅}£
    t = &s[1];
    £\ass{\s_{\mid 1}↦\t  *  \t_{\mid 1}↦\s  *  \s<\t  *  \brka(\t+1)  *  A=∅}£
    v = &s[0];
    £\ass{\s_{\mid 1}↦\t  *  \t_{\mid 1}↦\s  *  \s<\t  *  \v=\s  *  \brka(\t+1)  *  A=∅}£
    £\ass{\sblock\s\,\t\,\{\s+1 \ptoS 0\}  *  \t_{\mid 1}↦\s  *  \v=\s  *  \brka(\t+1)  *  A=∅}£
    £\ass{∃B_1,B_2.\block^*\s\,\v\,B_1  *  \block^*\v\,\t\,B_2  *  \t_{\mid 1}↦\s {}\\{}
    *  A = (B_1 \uplus B_2)^{\sf a}  *  \brka(\t+1)  *  A=∅}£
    £\ass{\arena A  *  A=∅}£
    £\ass{\arena A}£
  }
  £\ass{∃B_1,B_2.\block^*\s\,\v\,B_1  *  \block^*\v\,\t\,B_2  *  \t_{\mid1} ↦ \s {}\\{} 
  *  A = (B_1 \uplus B_2)^{\sf a}  *  \brka(\t+1)}£
  nw=(nbytes+WORD+WORD-1)/WORD;
  £\ass{∃B_1,B_2.\block^*\s\,\v\,B_1  *  \block^*\v\,\t\,B_2  *  \t_{\mid1} ↦ \s {}\\{} 
  *  A = (B_1 \uplus B_2)^{\sf a}  *  \brka(\t+1)  *  \nw=1+\mathceil{\frac{\nbytes}{\WORD}}}£
  for(p=v; ; ) { 
    // Loop inv 1:
    £\ass{∃B_1,B_2.\block^*\s\,\p\,B_1  *  \block^*\p\,\t\,B_2  *  \t_{\mid1} ↦ \s {}\\{}
    *  A=(B_1\uplus B_2)^{\sf a}  *  \brka(\t+1) 
     *  \nw=1+\mathceil{\frac{\nbytes}{\WORD}}}£
    for(temp=0; ; ) {
      // Loop inv 2:
      £\ass{∃B_1,B_2.\block^*\s\,\p\,B_1  *  \block^*\p\,\t\,B_2  *  \t_{\mid1} ↦ \s {}\\{}
      *  A=(B_1\uplus B_2)^{\sf a}  *  \brka(\t+1)
       *  \nw=1+\mathceil{\frac{\nbytes}{\WORD}}}£
      if(!testbusy(p->ptr)) {
        £\ass{∃B_1,B_2,q.\block^*\s\,\p\,B_1 *  \ublock \p\,q\,\{\p+1\ptoU q-\p-1\}  *  \block^*q\,\t\,B_2 {}\\{}
        *  \t_{\mid1} ↦ \s  *  A=(B_1\uplus B_2)^{\sf a}  *  \brka(\t+1) 
         *  \nw=1+\mathceil{\frac{\nbytes}{\WORD}}}£ 
	while(!testbusy((q=p->ptr)->ptr)) {
          £\ass{∃B_1,B_2,r.\block^*\s\,\p\,B_1 *  \ublock \p\,\q\,\{\p+1\ptoU \q-\p-1\} {}\\{}
          *  \ublock \q\,r\,\{\q+1\ptoU r-\q-1\}  *  \block^*r\,\t\,B_2  *  \t_{\mid1} ↦ \s {}\\{}
          *  A=(B_1\uplus B_2)^{\sf a}  *  \brka(\t+1)  *  \nw=1+\mathceil{\frac{\nbytes}{\WORD}}}£ 
	  p->ptr = q->ptr; // coalesce consecutive free blocks
          £\ass{∃B_1,B_2,r.\block^*\s\,\p\,B_1 *  \ublock \p\,r\,\{\p+1\ptoU r-\p-1\}  *  \block^*r\,\t\,B_2 {}\\{}
          *  \t_{\mid1} ↦ \s  *  A=(B_1\uplus B_2)^{\sf a}  *  \brka(\t+1) 
           *  \nw=1+\mathceil{\frac{\nbytes}{\WORD}}}£ 
 	}
        £\ass{∃B_1,B_2.\block^*\s\,\p\,B_1 *  \ublock \p\,\q\,\{\p+1\ptoU \q-\p-1\}  *  \block^*\q\,\t\,B_2 {}\\{}
        *  \t_{\mid1} ↦ \s  *  A=(B_1\uplus B_2)^{\sf a}  *  \brka(\t+1) 
         *  \nw=1+\mathceil{\frac{\nbytes}{\WORD}}}£ 
	if(q>=p+nw && p+nw>=p) {
          £\ass{∃B_1,B_2.\block^*\s\,\p\,B_1 *  \ublock \p\,\q\,\{\p+1\ptoU \q-\p-1\} {}\\{}
          *  \block^*\q\,\t\,B_2  *  \t_{\mid1} ↦ \s  *  A=(B_1\uplus B_2)^{\sf a}  *  \brka(\t+1) {}\\{}
          *  \nw=1+\mathceil{\frac{\nbytes}{\WORD}}  *  \q\geq\p+\nw}£ 
	  goto found;
          £\ass{\false}£
        } 
      }
      // p's block is unavailable / too small,
      // or p points to the top of the arena
      £\ass{∃B_1,B_2.\block^*\s\,\p\,B_1  *  \block^*\p\,\t\,B_2  *  A=(B_1\uplus B_2)^{\sf a} {}\\{}
      *  \t_{\mid1} ↦ \s  *  \brka(\t+1)   *  \nw=1+\mathceil{\frac{\nbytes}{\WORD}}}£
      q = p;
      £\ass{∃B_1,B_2.\block^*\s\,\q\,B_1  *  \block^*\q\,\t\,B_2  *  A=(B_1\uplus B_2)^{\sf a} {}\\{}
      *  \t_{\mid1} ↦ \s  *  \brka(\t+1)   *  \nw=1+\mathceil{\frac{\nbytes}{\WORD}}  *  \q=\p}£
      £\ass{∃B_1,B_2.\block^*\s\,\q\,B_1  *  \block^*\q\,\t\,B_2  *  A=(B_1\uplus B_2)^{\sf a} {}\\{}
      *  \t_{\mid1} ↦ \s  *  \brka(\t+1)   *  \nw=1+\mathceil{\frac{\nbytes}{\WORD}}  *  \q=\p}£
      p = clearbusy(p->ptr);
      £\ass{((∃B_1,B_2,τ.\block^*\s\,\q\,B_1  *  \block \q\,\p\,\{\q+1↦_τ\p-\q-1\} {}\\{} 
      *  \block^*\p\,\t\,B_2  *  A=(B_1\uplus \{\q+1↦_τ\p-\q-1\} \uplus B_2)^{\sf a}) \\{}
      ∨ (∃B.\block^*\s\,\q\,B  *  A=B^{\sf a}  *  \q=\t  *  \p=\s)) {}\\{}
      *  \t_{\mid1} ↦ \s  *  \brka(\t+1)  *  \nw=1+\mathceil{\frac{\nbytes}{\WORD}} }£
      if(p>q) {
        £\ass{∃B_1,B_2,τ.\block^*\s\,\q\,B_1  *  \block \q\,\p\,\{\q+1↦_τ\p-\q-1\} {}\\{}
        *  \block^*\p\,\t\,B_2  *  A = (B_1\uplus \{\q+1↦_τ\p-\q-1\} \uplus B_2)^{\sf a} {}\\{}
        *  \t_{\mid1} ↦ \s  *  \brka(\t+1) *  \nw=1+\mathceil{\frac{\nbytes}{\WORD}}}£
      } else if(q!=t || p!=s) {
        £\ass{∃B.\block^*\s\,\q\,B  *  \t_{\mid1} ↦ \s  *  A=B^{\sf a}  *  \brka(\t+1) {}\\{}
        *  \nw=1+\mathceil{\frac{\nbytes}{\WORD}}  *  \q=\t  *  \p=\s  *  (\q≠\t ∨ \p≠\s)}£
        £\ass{\false}£
	return 0; // unreachable
        £\ass{\false}£
      } else if(++temp>1) {
        £\ass{∃B.\block^*\s\,\q\,B  *  \t_{\mid1} ↦ \s   *  A=B^{\sf a}  *  \brka(\t+1) {}\\{}
        *  \nw=1+\mathceil{\frac{\nbytes}{\WORD}}  *  \q=\t  *  \p=\s}£
	break; // jump to [Extend arena]
        £\ass{\false}£
      } 
      // Reestablish loop inv 2:     
      £\ass{∃B_1,B_2.\block^*\s\,\p\,B_1  *  \block^*\p\,\t\,B_2  *  A=(B_1\uplus B_2)^{\sf a}  {}\\{} 
      *  \t_{\mid1} ↦ \s  *  \brka(\t+1)
       *  \nw=1+\mathceil{\frac{\nbytes}{\WORD}}}£
    }
    // We never exit the loop `normally' (because the non-existent 
    // test condition never fails). We only reach this point by 
    // breaking.
    // [Extend arena]:
    £\ass{∃B.\block^*\s\,\t\,B  *  \t_{\mid1} ↦ \s 
     *  A=B^{\sf a} {}\\{}
    *  \brka(\t+1)  *  \nw=1+\mathceil{\frac{\nbytes}{\WORD}}  *  \p=\s}£
    temp = ((nw+BLOCK/WORD)/(BLOCK/WORD))*(BLOCK/WORD); 
    £\ass{∃B.\block^*\s\,\t\,B  *  \t_{\mid1} ↦ \s 
     *  A=B^{\sf a}  *  \brka(\t+1) {}\\{}
    *  \nw=1+\mathceil{\frac{\nbytes}{\WORD}}  *  \p=\s  *  \temp>\nw}£
    q = (struct store *)sbrk(0);
    // note that £$\brka(\q) ⟹ \brka(\t+1)$£ because £$\q≥\t+1$£
    £\ass{∃B.\block^*\s\,\t\,B  *  \t_{\mid1} ↦ \s  *  A=B^{\sf a}  *  \brka(\t+1) {}\\{}
    *  \nw=1+\mathceil{\frac{\nbytes}{\WORD}}  *  \p=\s  *  \temp>\nw  *  \q≥\t+1}£
    if(q + temp < q) {
      £\ass{\false}£ // integer overflows aren't modelled
      return 0; 
      £\ass{\false}£      
    }
    £\ass{∃B.\block^*\s\,\t\,B  *  \t_{\mid1} ↦ \s  *  A=B^{\sf a}  *  \brka(\t+1) {}\\{}
    *  \nw=1+\mathceil{\frac{\nbytes}{\WORD}}  *  \p=\s  *  \temp>\nw  *  \q≥\t+1}£
    q = (struct store *)sbrk(temp * WORD);
    £\ass{∃B.\block^*\s\,\t\,B  *  \t_{\mid1} ↦ \s  *  A=B^{\sf a} 
     *  \nw=1+\mathceil{\frac{\nbytes}{\WORD}} {}\\{} *  \p=\s  *  \temp>\nw  *  ((\brka(\t+1)  *  \q=-1) {}\\{}
    ∨ (\brka(\q+\temp)  *  \t+1≤\q  *  \iterstar[\temp-1]{i=0}.\q+i↦\underscore))}£
    if((INT)q == -1) {
      £\ass{∃B.\block^*\s\,\t\,B  *  \t_{\mid1} ↦ \s  *  A=B^{\sf a}  *  \brka(\t+1)}£
      v = s; // line added to fix bug
      £\ass{∃B_1,B_2.\block^*\s\,\v\,B_1  *  \block^*\v\,\t\,B_2  * \t_{\mid1} ↦ \s {}\\{}
      *  A=(B_1\uplus B_2)^{\sf a}  *  \brka(\t+1)}£
      £\ass{\arena A}£
      £\ass{(\arena A  *  \ret=0) [0/\ret]}£
      return 0;
      £\ass{\false}£
    }
    £\ass{∃B.\block^*\s\,\t\,B  *  \t_{\mid1} ↦ \s  *  A=B^{\sf a} 
     *  \nw=1+\mathceil{\frac{\nbytes}{\WORD}}  *  \p=\s {}\\{}
    *  \temp>\nw  *  \brka(\q+\temp)  *  \t+1≤\q  *  \iterstar[\temp-1]{i=0}.\q+i↦\underscore}£
    t->ptr = q;
    £\ass{∃B.\block^*\s\,\t\,B  *  \t ↦ \q  *  A=B^{\sf a} 
     *  \nw=1+\mathceil{\frac{\nbytes}{\WORD}}  *  \p=\s {}\\{}
    *  \temp>\nw  *  \brka(\q+\temp)  *  \t+1≤\q  *  \iterstar[\temp-1]{i=0}.\q+i↦\underscore}£
    if(q!=t+1) {
      £\ass{∃B.\block^*\s\,\t\,B  *  \t ↦ \q  *  A=B^{\sf a} 
       *  \nw=1+\mathceil{\frac{\nbytes}{\WORD}}  *  \p=\s {}\\{}
      *  \temp>\nw  *  \brka(\q+\temp)  *  \t+1<\q  *  \iterstar[\temp-1]{i=0}.\q+i↦\underscore}£
      t->ptr = setbusy(t->ptr);
      £\ass{∃B.\block^*\s\,\t\,B  *  \t_{\mid 1} ↦ \q  *  A=B^{\sf a}
       *  \nw=1+\mathceil{\frac{\nbytes}{\WORD}}  *  \p=\s {}\\{}
      *  \temp>\nw  *  \brka(\q+\temp)  *  \t+1<\q  *  \iterstar[\temp-1]{i=0}.\q+i↦\underscore}£
      £\ass{∃B.\block^*\s\,\t\,B  *  \sblock \t\,\q\,\{\t+1\ptoS\q-\t-1\}  *  A=B^{\sf a} 
       *  \nw=1+\mathceil{\frac{\nbytes}{\WORD}} {}\\{} *  \p=\s  *  \temp>\nw  *  \brka(\q+\temp) 
       *  \iterstar[\temp-1]{i=0}.\q+i↦\underscore}£
    }
    // t is either a ublock of size 0 or an sblock
    £\ass{∃B,τ.\block^*\s\,\t\,B  *  \block \t\,\q\,\{\t+1↦_τ\q-\t-1\}  *  A=B^{\sf a} 
     *  \nw=1+\mathceil{\frac{\nbytes}{\WORD}} {}\\{} *  \p=\s  *  \temp>\nw  *  \brka(\q+\temp) 
     *  \iterstar[\temp-1]{i=0}.\q+i↦\underscore}£
    // B swallows the block at t. A=B^a still holds because 
    // the block at t isn't allocated.
    £\ass{∃B.\block^*\s\,\q\,B  *  A=B^{\sf a}
     *  \nw=1+\mathceil{\frac{\nbytes}{\WORD}}  *  \p=\s  *  \temp>\nw {}\\{} *  \brka(\q+\temp) 
     *  \q↦\underscore  *  \iterstar[\q+\temp-2]{i=\q+1}.i↦\underscore  *  (\q+\temp-1) ↦ \underscore}£
    t = q->ptr = q+temp-1;
    £\ass{∃B.\block^*\s\,\q\,B  *  A=B^{\sf a} 
     *  \nw=1+\mathceil{\frac{\nbytes}{\WORD}}  *  \p=\s {}\\{}  *  \brka(\t+1) 
     *  \q<\t  *  \q↦\t  *  \iterstar[\t-1]{i=\q+1}.i↦\underscore  *  \t ↦ \underscore}£
    £\ass{∃B.\block^*\s\,\q\,B  *  A=B^{\sf a} 
     *  \nw=1+\mathceil{\frac{\nbytes}{\WORD}}  *  \p=\s {}\\{}  *  \brka(\t+1) 
     *  \ublock\q\,\t\,\{\q+1\ptoU \t-\q-1\}  *  \t ↦ \underscore}£
    // B swallows the block at q. A=B^a still holds because 
    // the block at q isn't allocated.
    £\ass{∃B.\block^*\s\,\t\,B  *  A=B^{\sf a} 
     *  \nw=1+\mathceil{\frac{\nbytes}{\WORD}} {}\\{} *  \p=\s  *  \brka(\t+1) 
     *  \t ↦ \underscore}£
    t->ptr = setbusy(s);
    // reestablish loop inv 1:
    £\ass{∃B_1,B_2.\block^*\s\,\p\,B_1  *  \block^*\p\,\t\,B_2  *  \t_{\mid 1} ↦ \s {}\\{}
    *  A=(B_1\uplus B_2)^{\sf a}  *  \brka(\t+1)  *  \nw=1+\mathceil{\frac{\nbytes}{\WORD}}}£
  }
  £\ass{\false}£
  found:
  £\ass{∃B_1,B_2.\block^*\s\,\p\,B_1 *  \ublock \p\,\q\,\{\p+1\ptoU \q-\p-1\} {}\\{}
  *  \block^*\q\,\t\,B_2  *  \t_{\mid1} ↦ \s  *  A = (B_1 \uplus B_2)^{\sf a}  *  \brka(\t+1) {}\\{}
  *  \nw=1+\mathceil{\frac{\nbytes}{\WORD}}  *  \q\geq\p+\nw}£ 
  v = p+nw;
  £\ass{∃B_1,B_2.\block^*\s\,\p\,B_1 *  \p<\q  *  \p↦\q  *  \iterstar[\q-1]{i=\p+1}.i↦\underscore {}\\{}
  *  \block^*\q\,\t\,B_2  *  \t_{\mid1} ↦ \s  *  A = (B_1 \uplus B_2)^{\sf a}  *  \brka(\t+1) {}\\{}
  *  \nw=1+\mathceil{\frac{\nbytes}{\WORD}}  *  \q\geq\v  *  \v=\p+\nw}£ 
  if (q>v) {
    £\ass{∃B_1,B_2.\block^*\s\,\p\,B_1 *  \p<\q  *  \p↦\q  *  \iterstar[\q-1]{i=\p+1}.i↦\underscore {}\\{}
    *  \block^*\q\,\t\,B_2  *  \t_{\mid1} ↦ \s  *  A = (B_1 \uplus B_2)^{\sf a}  *  \brka(\t+1) {}\\{}
    *  \nw=1+\mathceil{\frac{\nbytes}{\WORD}} *  \q>\v  *  \v=\p+\nw}£ 
    v->ptr = p->ptr;
    £\ass{∃B_1,B_2.\block^*\s\,\p\,B_1 *  \p↦\q  *  \iterstar[\v-1]{i=\p+1}.i↦\underscore {}\\{}
    *  \ublock \v\,\q\,\{(\v+1)\ptoU(\q-\v-1)\} {}\\{} 
    *  \block^*\q\,\t\,B_2  *  \t_{\mid1} ↦ \s  * A = (B_1 \uplus B_2)^{\sf a}  *  \brka(\t+1) {}\\{}
    *  \nw=1+\mathceil{\frac{\nbytes}{\WORD}}  *  \v=\p+\nw}£ 
    £\ass{∃B_1,B_2.\block^*\s\,\p\,B_1 *  \p↦\q  *  \iterstar[\v-1]{i=\p+1}.i↦\underscore 
     *  \block^*\v\,\t\,B_2 {}\\{} *  \t_{\mid1} ↦ \s  *  A = (B_1 \uplus B_2)^{\sf a} 
     *  \brka(\t+1)  *  \nw=1+\mathceil{\frac{\nbytes}{\WORD}}  *  \v=\p+\nw}£ 
  }
  £\ass{∃B_1,B_2.\block^*\s\,\p\,B_1 *  \p↦\q  *  \iterstar[\v-1]{i=\p+1}.i↦\underscore 
   *  \block^*\v\,\t\,B_2 {}\\{} *  \t_{\mid1} ↦ \s  *  A = (B_1 \uplus B_2)^{\sf a}
   *  \brka(\t+1)  *  \nw=1+\mathceil{\frac{\nbytes}{\WORD}}  *  \v=\p+\nw}£
  p->ptr = setbusy(v);
  £\ass{∃B_1,B_2.\block^*\s\,\p\,B_1 *  \p_{\mid 1}↦\v  *  \iterstar[\v-1]{i=\p+1}.i↦\underscore 
   *  \block^*\v\,\t\,B_2 {}\\{} *  \t_{\mid1} ↦ \s  *  A = (B_1 \uplus B_2)^{\sf a}
   *  \brka(\t+1)  *  \nw=1+\mathceil{\frac{\nbytes}{\WORD}}  *  \v=\p+\nw}£
  £\ass{∃B_1,B_2.\block^*\s\,\p\,B_1  *  \ablock\p\,\v\,\{\p+1\ptoA\nw-1\} 
   *  \block^*\v\,\t\,B_2 {}\\{} *  \t_{\mid1} ↦ \s  *  A = (B_1 \uplus B_2)^{\sf a} 
   *  \brka(\t+1)  *  \nw=1+\mathceil{\frac{\nbytes}{\WORD}} {}\\{}
  *  \p_{\mid 1}\pto[.5]\v  *  \iterstar[\v-1]{i=\p+1}.i↦\underscore   *  \v=\p+\nw}£
  // use lemma to deduce that B1 and p+1 are disjoint
  £\ass{∃B_1,B_2.\block^*\s\,\v\,B_1  *  \block^*\v\,\t\,B_2  *  \t_{\mid1} ↦ \s {}\\{}
  *  A \uplus \{\p+1↦ \mathceil{\frac{\nbytes}{\WORD}}\} = (B_1\uplus B_2)^{\sf a} {}\\{}
  *  \brka(\t+1)  *  \p_{\mid 1}\pto[.5]\p+\mathceil{\frac{\nbytes}{\WORD}}+1 {}\\{} 
  * \iterstar[\mathceil{\nbytes/\WORD}-1]{i=0}.\p+1+i↦\underscore}£
  £\ass{(\arena(A\uplus\{\ret↦\mathceil{\frac{\nbytes}{\WORD}}\}) 
   *  \iterstar[\mathceil{\nbytes/\WORD}-1]{i=0}.\ret+i↦\underscore {}\\{}
   *  (\ret-1)_{\mid 1}\pto[.5]\ret+\mathceil{\frac{\nbytes}{\WORD}}) [\p+1/\ret]}£
  return((char *)(p+1));
  £\ass{\false}£
}
£\ass{\left(\ml[c]{\arena(A\uplus\{\ret↦\mathceil{\nbytes/\WORD}\}) {}\\{}
*  \iterstar[\mathceil{\nbytes/\WORD}-1]{i=0}.\ret+i↦\underscore {}\\{}
*  (\ret-1)_{\mid 1}\pto[.5]\ret+\mathceil{\nbytes/\WORD}}\right) ∨ (\arena\,A  *  \ret=0)}£
// end region update
£\ass{\left(\ml[c]{\boxed{\arena(A\uplus\{\ret↦\mathceil{\nbytes/\WORD}\})} {}\\{}
*  \iterstar[\mathceil{\nbytes/\WORD}-1]{i=0}.\ret+i↦\underscore {}\\{}
*  (\ret-1)_{\mid 1}\pto[.5]\ret+\mathceil{\nbytes/\WORD}}\right) ∨ (\boxed{\arena\,A}  *  \ret=0)}£
// end existential
£\ass{\left(\ml[c]{\boxed{∃A.\arena(A\uplus\{\ret↦\mathceil{\nbytes/\WORD}\})} {}\\{}
*  \iterstar[\mathceil{\nbytes/\WORD}-1]{i=0}.\ret+i↦\underscore {}\\{}
*  (\ret-1)_{\mid 1}\pto[.5]\ret+\mathceil{\nbytes/\WORD}}\right) ∨ (\boxed{∃A.\arena\,A}  *  \ret=0)}£
// note that £$∃A.\arena(A\uplus\{\ret↦\mathceil{\nbytes/\WORD}\})$£ implies £$∃A.\arena(A)$£
£\ass{\left(\ml[c]{\boxed{∃A.\arena\,A} {}\\{}
*  \boxed{∃A.\arena(A\uplus\{\ret↦\mathceil{\nbytes/\WORD}\})} {}\\{}
*  \iterstar[\mathceil{\nbytes/\WORD}-1]{i=0}.\ret+i↦\underscore {}\\{}
*  (\ret-1)_{\mid 1}\pto[.5]\ret+\mathceil{\nbytes/\WORD}}\right) ∨ (\boxed{∃A.\arena\,A}  *  \ret=0)}£
£\ass{\left(\ml[c]{\boxed{∃A.\uninit A  ∨  \arena\,A} {}\\{}
*  \boxed{∃A.\arena(A\uplus\{\ret↦\mathceil{\nbytes/\WORD}\})} {}\\{}
*  \iterstar[\mathceil{\nbytes/\WORD}-1]{i=0}.\ret+i↦\underscore {}\\{}
*  (\ret-1)_{\mid 1}\pto[.5]\ret+\mathceil{\nbytes/\WORD}}\right) ∨ (\boxed{∃A.\uninit A  ∨  \arena\,A}  *  \ret=0)}£
£\ass{\left(\ml[c]{\anArena {}\\{}
*  \token(\ret,\mathceil{\nbytes/\WORD}) {}\\{}
*  \iterstar[\mathceil{\nbytes/\WORD}-1]{i=0}.\ret+i↦\underscore}\right) ∨ (\anArena  *  \ret=0)}£
£\ass{\anArena  *  \ml[t]{ ((\token\ret\,\mathceil{\nbytes/\WORD}  *  \iterstar[\mathceil{\nbytes/\WORD}-1]{i=0}.\ret+i↦\underscore)  ∨  \ret = 0 )}}£
\end{lstlisting}\ \\



\subsubsection*{Verification of free routine}

\begin{lstlisting}
free(register char *ap) 
£\ass{\anArena  *  ∃n.\token\,{\tt ap}\,n  *  \iterstar[n-1]{i=0}.({\tt ap}+i)↦\underscore}£
£\ass{∃n.\boxed{∃A.\uninit A  ∨  \arena A}  *  \boxed{∃A.arena(A \uplus \{{\tt ap}↦n\})} 
 *  ({\tt ap}-1)_{\mid 1}\pto[.5]({\tt ap}+n) {}\\{}
*  \iterstar[n-1]{i=0}.({\tt ap}+i)↦\underscore}£
£\ass{∃n.\boxed{∃A.arena(A \uplus \{{\tt ap}↦n\})} 
 *  ({\tt ap}-1)_{\mid 1}\pto[.5]({\tt ap}+n)  *  \iterstar[n-1]{i=0}.{\tt ap}+i↦\underscore}£
//begin existential
£\ass{\boxed{arena(A \uplus \{{\tt ap}↦n\})} 
 *  ({\tt ap}-1)_{\mid 1}\pto[.5]({\tt ap}+n)  *  \iterstar[n-1]{i=0}.{\tt ap}+i↦\underscore}£
//begin "Free x" action
{
  £\ass{arena(A \uplus \{{\tt ap}↦n\}) 
   *  ({\tt ap}-1)_{\mid 1}\pto[.5]({\tt ap}+n)  *  \iterstar[n-1]{i=0}.{\tt ap}+i↦\underscore}£
  £\ass{∃B_1,B_2.\block^*\,\s\,\v\,B_1  *  \block^*\,\v\,\t\,B_2  *  A \uplus \{{\tt ap}↦n\} = (B_1 \uplus B_2)^{\sf a} 
   *  \t_{\mid1} ↦ \s {}\\{} *  \brka(\t+1)  *  ({\tt ap}-1)_{\mid 1}\pto[.5]({\tt ap}+n) 
   *  \iterstar[n-1]{i=0}.{\tt ap}+i↦\underscore}£
  // use lemma to deduce that B1 and B2 are disjoint
  £\ass{∃B.\block^*\,\s\,\t\,B  *  A \uplus \{{\tt ap}↦n\} = B^{\sf a} 
   *  \t_{\mid1} ↦ \s {}\\{} *  \brka(\t+1)  *  ({\tt ap}-1)_{\mid 1}\pto[.5]({\tt ap}+n) 
   *  \iterstar[n-1]{i=0}.{\tt ap}+i↦\underscore}£
  // note that £$\{x \ptoA n\} ∈ B$£ implies £$∃B_1,B_2.B = B_1 \uplus \{x \ptoA n\} \uplus B_2$£
  £\ass{∃B_1,B_2.\block^*\,\s\,(\ap-1)\,B_1  *  \ablock\,(\ap-1)\,(\ap+n)\,\{{\tt ap}\ptoA n\} {}\\{}
  *  \block^*\,(\ap+n)\,\t\,B_2  *  A \uplus \{{\tt ap}↦n\} = (B_1\uplus \{{\tt ap}\ptoA   n\}\uplus B_2)^{\sf a}
   *  \t_{\mid1} ↦ \s {}\\{} *  \brka(\t+1)  *  ({\tt ap}-1)_{\mid 1}\pto[.5]({\tt ap}+n) 
   *  \iterstar[n-1]{i=0}.{\tt ap}+i↦\underscore}£
  // by cancellativity of £$\uplus$£:
  £\ass{∃B_1,B_2.\block^*\,\s\,(\ap-1)\,B_1  *  \ablock\,(\ap-1)\,(\ap+n)\,\{{\tt ap}\ptoA n\} {}\\{}
  *  \block^*\,(\ap+n)\,\t\,B_2  *  A = (B_1\uplus B_2)^{\sf a}
   *  \t_{\mid1} ↦ \s {}\\{} *  \brka(\t+1)  *  ({\tt ap}-1)_{\mid 1}\pto[.5]({\tt ap}+n) 
   *  \iterstar[n-1]{i=0}.{\tt ap}+i↦\underscore}£
  register struct store *p = (struct store *)ap;
  v = --p;
  £\ass{∃B_1,B_2.\block^*\,\s\,\p\,B_1  *  \ablock\,\p\,(\p+1+n)\,\{\p+1\ptoA n\} {}\\{}
  *  \block^*\,(\p+1+n)\,\t\,B_2  *  A = (B_1\uplus B_2)^{\sf a}
   *  \t_{\mid1} ↦ \s {}\\{} *  \brka(\t+1)  *  \p_{\mid 1}\pto[.5](\p+1+n) 
   *  \iterstar[n-1]{i=0}.\p+1+i↦\underscore   *  \p=\v}£
  £\ass{∃B_1,B_2.\block^*\,\s\,\p\,B_1  *  \p_{\mid 1} ↦ \p+1+n  *  \iterstar[n-1]{i=0}.\p+1+i↦\underscore {}\\{}
  *  \block^*\,(\p+1+n)\,\t\,B_2  *  A = (B_1\uplus B_2)^{\sf a}
   *  \t_{\mid1} ↦ \s  *  \brka(\t+1)  *  \p=\v}£
  p->ptr = clearbusy(p->ptr);
  £\ass{∃B_1,B_2.\block^*\,\s\,\p\,B_1  *  \p ↦ \p+1+n  *  \iterstar[n-1]{i=0}.\p+1+i↦\underscore {}\\{}
  *  \block^*\,(\p+1+n)\,\t\,B_2  *  A = (B_1\uplus B_2)^{\sf a}
   *  \t_{\mid1} ↦ \s  *  \brka(\t+1)  *  \p=\v}£
  £\ass{∃B_1,B_2.\block^*\,\s\,\p\,B_1  *  \ublock\p\,(\p+1+n)\,\{\p+1\ptoU n\} {}\\{}
  *  \block^*\,(\p+1+n)\,\t\,B_2  *  A = (B_1\uplus B_2)^{\sf a}
   *  \t_{\mid1} ↦ \s  *  \brka(\t+1)  *  \p=\v}£
  // use lemma to deduce that p and B2 are disjoint
  £\ass{∃B_1,B_2.\block^*\,\s\,\v\,B_1  *  \block^*\,\v\,\t\,B_2  *  A = (B_1 \uplus B_2)^{\sf a}  *  \t_{\mid1} ↦ \s  *  \brka(\t+1)}£
  £\ass{\arena A}£
}
//end "Free x" action
£\ass{\boxed{\arena A}}£
//end existential
£\ass{\boxed{∃A.\arena A}}£
£\ass{\boxed{∃A.\uninit A  ∨  \arena A}}£
£\ass{\anArena}£
\end{lstlisting}\ \\




\end{document}
